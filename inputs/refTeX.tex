\section{Présentation de REF\TeX}
\label{sec:reftex}

\vfill
\begin{itemize}
\item REF\TeX\ est un mode \emph{mineur} d'Emacs qui peut s'ajouter à AUC\TeX.
\item Il a pour fonction de gérer les références de tout type\ldots et notamment
  les références bibliographiques.
\item Pour le lancer, on tape \verb1M-x reftex-mode1 dans Emacs.
\item Mais on peut aussi le charger automatiquement dès qu'AUC\TeX\ est lui-même
  lancé. Pour celà, il faut entrer les lignes suivantes dans le fichier de
  démarrage d'Emacs (\verb1.emacs1):
  \begin{itemize}
%  \item \verb1(require 'tex-site)1
  \item \verb1(add-hook 'LaTeX-mode-hook 'turn-on-reftex)1
  \item \verb1(setq reftex-plug-into-AUCTeX t)1
  \end{itemize}
\end{itemize}
\vfill

% '(reftex-default-bibliography (quote ("/home/olivier/texmf/bibtex/bib/olivier-perso.bib" "/home/olivier/texmf/bibtex/bib/speech-science.bib")))

Si le mode REF\TeX\ est bien lancé, vous devriez voir un menu \og Ref
\fg apparaître dans la barre de menu d'Emacs.

\section{Gestion des références bibliographiques}

Le menu Ref $\Rightarrow$ Cite vous permet de faire une recherche dans
votre base de données bibliographique.

Si vous sélectionnez une entrée dans celles proposées, il insère la
commande nécessaire à l'endroit où se trouve le curseur.


\section{Gestion des renvois}

La gestion des renvois fonctionne de la même manière : vous pouvez
utiliser Ref $\Rightarrow$ Label pour insérer un label, et Ref
$\Rightarrow$ Ref pour insérer un renvoi. Dans ce dernier cas, Emacs
vous proposera de sélectionner le type de référence (section,
équation, figure, tableau\ldots --un appui sur la barre espace fera
une recherche sur toutes les sources possible), puis vous présentera
une liste des éléments vers lesquels vous pouvez produire un renvoi.
