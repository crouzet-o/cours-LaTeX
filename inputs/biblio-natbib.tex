\section{En-tête du document}

L'extension \emph{natbib} permet de générer des références
bibliographiques de type \emph{auteur (année)}. Pour utiliser
l'extension \emph{natbib}, vous appelerez cette extension dans
l'en-tête du document :


\begin{verbatim}
\usepackage[longnamesfirst,round]{natbib}
\end{verbatim}

Les options (facultatives) présentées ici ont pour fonction (1) de
forcer l'affichage de la liste intégrale des auteurs au premier appel
et (2) de présenter l'année de publication entre parenthèses.







\section{Appel des références bibliographiques}

Afin d'utiliser au mieux l'extension natbib, il convient de prendre
l'habitude d'utiliser différentes formes de \verb1\cite1 afin de
contrôler précisément le type de renvoi bibliographique :


\begin{itemize}
\item \verb1\cite{} = \citet{}1 (c'est la forme par défaut, cite le
  nom d'auteur(s) \emph{dans le texte} et affiche l'année entre
  parenthèses);

\item \verb1\citep{}1 (tout entre parenthèses, une virgule sépare la
  liste des auteurs de l'année);

\item \verb1\citeauthor{}1 (auteur seul);

\item \verb1\citeyear{}1 (année seule);

\item \verb1\cite[chap.~2]{}1 (commentaire après la citation);

\item \verb1\cite[cf.][]{}1 (commentaire avant la citation);

\item \verb1\cite[cf.][chap.~2]{}1 (commentaires avant \emph{et} après
  la citation);

\item \verb1\cite*{}1 (force la citation de tous les auteurs au lieu
  de et al. pour cet appel, toutes les autres formes affichant les
  noms d'auteurs peuvent être suivies d'une astérisque pour produire
  le même effet).
\end{itemize}






\section{Formatage de la bibliographie}

Comme d'habitude, vous introduirez --à l'endroit où vous souhaitez
faire apparaître votre liste de références bibliographiques-- les
commandes suivantes :

\begin{verbatim}
\bibliography{nom-de-votre-base}
\bibliographystyle{abbrvnat}
\end{verbatim}

Pour le reste, le fonctionnement est similaire. L'extension
\emph{natbib} fournit des formats de bibliographies permettant de
remplacer les formats standards (les noms sont similaires mais se
terminent par -nat).

Vous pouvez aussi utiliser le format \verb1apaformat.bst1 que je tiens
à votre disposition et qui reproduit des exigences communes dans de
nombreux sous-domaines des sciences humaines et sociales (type
\emph{American Psychological Association}).


%%% Local Variables: 
%%% mode: latex
%%% TeX-master: "~/home/teaching/cours/sdl/M2/LaTeX/slides/recueil-initiation-LaTeX"
%%% End: 
