


\section{Principaux environnements}

\begin{itemize}
\item \emph{abstract}

  \verb1\begin{abstract} ... \end{abstract}1
  
  Précédé éventuellement de \verb1\selectlanguage{francais|english|...}1 pour
  sélectionner la langue. On peut insérer plusieurs environnements
  \emph{abstract} utilisant une langue différente.

\item justification du texte (par défaut, le texte standard est \og
  justifié \fg)

  \begin{itemize}
  \item \emph{center} (comme son nom l'indique, permet de centrer les
    objets (texte, graphiques, tableaux\ldots). Il existe une
    \emph{déclaration} équivalente --tous les environnements ne
    corrspondent pas nécessairement à une déclaration-- :
    \verb1\centering1)

    \verb1\begin{center} ... \end{center}1

  \item \emph{flushright} (alignement à droite, déclaration
    équivalente : \verb1\raggedleft1 -- attention \emph{ragged} doit
    être compris comme : \og je pousse le texte depuis ce côté-là \fg,
    ce qui explique que l'environnement \emph{flushright} corresponde
    à la déclaration \emph{raggedleft}).
  \item \emph{flushleft} (alignement à gauche, déclaration équivalente
    : \verb1\raggedright1).
  \end{itemize}

\item citations
  \begin{itemize}
  \item \emph{quote} (citation d'un seul paragraphe)
    \begin{exemple}[H]
      \caption{Citation avec l'environnement \emph{quote}}
\begin{verbatim}
\begin{quote}
  Ceci est un exemple de rédaction avec LaTeX. Vous pouvez bien
  évidemment le reproduire pour vous entraîner. Si vous rencontrez des
  difficultés (messages d'erreur par exemple), essayez de localiser la
  source de l'erreur en regardant les messages fournis par LaTeX et de
  corriger ensuite l'erreur.
\end{quote}
\end{verbatim}
    \end{exemple}
  \item \emph{quotation} (citation de plusieurs paragraphes)
    \begin{exemple}[H]
      \caption{Citation avec l'environnement \emph{quotation}}
\begin{verbatim}
\begin{quotation}
  Ceci est un exemple de rédaction avec LaTeX. Vous pouvez bien
  évidemment le reproduire pour vous entraîner. Si vous rencontrez des
  difficultés (messages d'erreur par exemple), essayez de localiser la
  source de l'erreur en regardant les messages fournis par LaTeX et de
  corriger ensuite l'erreur.

  Ceci est un exemple de rédaction avec LaTeX. Vous pouvez bien
  évidemment le reproduire pour vous entraîner. Si vous rencontrez des
  difficultés (messages d'erreur par exemple), essayez de localiser la
  source de l'erreur en regardant les messages fournis par LaTeX et de
  corriger ensuite l'erreur.
\end{quotation}
\end{verbatim}
    \end{exemple}
  \item \emph{verse} (pour transcrire des poèmes; chaque vers se
    termine par \verb1\\1 afin de passer au vers suivant)
    \begin{exemple}[H]
      \caption{Citation avec l'environnement \emph{verse}}
\begin{verbatim}
\begin{verse}
  Ceci est un exemple de rédaction avec LaTeX.\\
  Vous pouvez bien évidemment le reproduire pour vous entraîner.\\

  Si vous rencontrez des difficultés\\
  (messages d'erreur par exemple),\\
  essayez de localiser la source de l'erreur\\
  en regardant les messages fournis par LaTeX\\
  et de corriger ensuite l'erreur.
\end{verse}
\end{verbatim}
    \end{exemple}
  \end{itemize}

\item listes (\verb1\item1 au début de chaque élément)
  \begin{itemize}
  \item \emph{itemize} (liste non-numérotée)
    \begin{exemple}[H]
      \caption{Liste non-numérotée avec l'environnement \emph{itemize}}
\begin{verbatim}
\begin{itemize}
  \item Ceci est un exemple de rédaction avec LaTeX.
  \item Vous pouvez bien évidemment le reproduire pour vous entraîner.
  \item Si vous rencontrez des difficultés (messages d'erreur par
    exemple), essayez de localiser la source de l'erreur en regardant
    les messages fournis par LaTeX et de corriger ensuite l'erreur.
\end{itemize}
\end{verbatim}
    \end{exemple}
  \item \emph{enumerate} (liste numérotée)
    \begin{exemple}[H]
      \caption{Liste numérotée avec l'environnement \emph{enumerate}}
\begin{verbatim}
\begin{enumerate}
  \item Ceci est un exemple de rédaction avec LaTeX.
  \item Vous pouvez bien évidemment le reproduire pour vous entraîner.
  \item Si vous rencontrez des difficultés (messages d'erreur par
    exemple), essayez de localiser la source de l'erreur en regardant
    les messages fournis par LaTeX et de corriger ensuite l'erreur.
\end{enumerate}
\end{verbatim}
    \end{exemple}
  \item \emph{description} (définitions + terme à définir entre crochets)

    \verb1\item[terme à définir] Définition du terme.1

    \begin{exemple}[H]
      \caption{Liste de termes avec l'environnement \emph{description}}
\begin{verbatim}
\begin{description}
  \item[1er terme] Ceci est un exemple de rédaction avec LaTeX.
  \item[terme suivant] Vous pouvez bien évidemment le reproduire pour
    vous entraîner.
  \item[dernier terme] Si vous rencontrez des difficultés (messages
    d'erreur par exemple), essayez de localiser la source de l'erreur
    en regardant les messages fournis par LaTeX et de corriger ensuite
    l'erreur.
\end{description}
\end{verbatim}
    \end{exemple}
  \end{itemize}
\end{itemize}


\section{Quelques déclarations courantes}


\begin{itemize}
\item Changement de type de police :
  \begin{itemize}
  \item \verb1\rmfamily1 : Roman (= Serif, cf. Times)
  \item \verb1\sffamily1 : Sans Serif (cf. Helvetica)
  \item \verb1\ttfamily1 : Typewriter (cf. Courier) \emph{Particulièrement utile
      pour les colonnes de chiffres}
  \end{itemize}
\item Changement d'orientation des caractères :
  \begin{itemize}
  \item \verb1\slshape1 : Slanted (italique)
  \item \verb1\upshape1 : Droit (par défaut)
  \item \verb1\scshape1 : Small Caps (petites capitales)
  \end{itemize}
\item Graisse :
  \begin{itemize}
  \item \verb1\bfseries1 : Gras
  \end{itemize}
\item Alignement horizontal :
  \begin{itemize}
  \item \verb1\centering1 : Centré
  \item \verb1\raggedleft1 : Repoussé depuis la gauche (Aligné à droite)
  \item \verb1\raggedright1 : Repoussé depuis la droite (Aligné à gauche)
  \end{itemize}
\item Taille de caractères :
  \begin{itemize}
  \item \verb1\tiny1 : du plus petit
  \item \verb1\scriptsize1 : \ldots
  \item \verb1\footnotesize1 : \ldots
  \item \verb1\small1 : \ldots
  \item \verb1\normalsize1 : \ldots
  \item \verb1\large1 : \ldots
  \item \verb1\Large1 : \ldots
  \item \verb1\LARGE1 : \ldots
  \item \verb1\huge1 : \ldots
  \item \verb1\Huge1 : au plus grand
  \end{itemize}
\end{itemize}

Attention, la plupart de ces déclarations n'ont pas pour fonction
d'être utilisée dans le corps du document mais seulement lorsque l'on
commence à configurer l'en-tête du document avec des outils
spécifiques. Par exemple, on pourrait souhaiter que tous les titres de
section apparaissent en police Sans-Serif, Gras, gros caractères
droits. \emph{\'Evidemment}, vous n'utiliserez pas ces déclarations
pour chaque instruction \verb1\section{}1\ldots nous verrons, à la fin
du cours, comment on gère ça avec \LaTeX.

\section{Les annexes}

\begin{itemize}
\item Tout ce qui suit la déclaration \verb1\appendix1 est considéré comme
  faisant partie des annexes.
\item Après cette déclaration, les éléments de titre (\verb1\section{}1,
  \verb1\subsection{}1\ldots) seront numérotés différemment (en standard avec
  des lettres), ainsi que les numéros de pages (en standard avec des chiffres
  romains;
\item Le comportement de cette instruction dépend beaucoup de la
  classe de document utilisée (la classe article \emph{article}
  donnera un résultat différent de ce qu'on obtiendrait avec la classe
  \emph{book} ou \emph{report}).
\end{itemize}

\section{Table des matières}

\begin{itemize}
\item \verb1\tableofcontents1
\item Insère une table des matières à l'endroit où cette instruction apparaît.
\end{itemize}

On peut éventuellement faire précéder l'appel à \verb1\tableofcontents1 des instructions suivantes :

\begin{itemize}
\item {\bfseries \verb=\setcounter{tocdepth}{2}=} : Profondeur
  maxi pour intégration dans la TdM.
\item {\bfseries \verb=\setcounter{secnumdepth}{1}=} : Profondeur maxi
  pour affichage de la numérotation des pages dans la table des
  matières.
\end{itemize}

\section{Listes des graphiques et des tableaux}


\subsection{Liste des graphiques}

\begin{itemize}
\item \verb1\listoffigures1 (cf. p.\pageref{list:fig})
\item Insère une liste des graphiques (et des numéros de pages
  correspondants) à l'endroit où cette instruction apparaît.
\item Inutile tant que nous n'avons pas vu comment insérer des graphiques.
\end{itemize}


\subsection{Liste des tableaux}

\begin{itemize}
\item \verb1\listoftables1 (cf. p.\pageref{list:tab})
\item Insère une liste des tableaux (et des numéros de pages
  correspondants) à l'endroit où cette instruction apparaît.
\item Inutile tant que nous n'avons pas vu comment insérer des tableaux.
\end{itemize}



\section{Les renvois}

\begin{itemize}
\item \verb1\label{nom}1 : insère un repère dans le document et lui attribue un
  nom (qui doit être unique parmi tous les labels attribués dans le document).
  \begin{itemize}
  \item En général, on fait précéder les labels des graphiques de
    \emph{\selectlanguage{english}fig:} et ceux des tableaux de
    \emph{\selectlanguage{english}tab:} (mais c'est juste une convention
    d'usage), par exemple \verb1\label{fig:surface}1 pour un graphique
    représentant une surface.
  \end{itemize}
\item \verb1\ref{nom}1 : effectue un renvoi vers le numéro [de la section, du
  tableau, du graphique, \ldots] qui correspond au repère \emph{nom}.
  \begin{itemize}
  \item Pour les tableaux, les graphiques (flottants, sinon ils ne disposent pas
    d'un numéro), il convient de situer l'appel \verb1\label{fig:surface}1 à
    l'intérieur de l'environnement (figure ou table), la position exacte importe
    peu.
  \item Pour les sections, sous-sections\ldots on insère en général le label
    juste après la section concernée.

\begin{verbatim}
\section{Un titre de section}
\label{théorie_Z}
\end{verbatim}

  \end{itemize}
\item \verb1\pageref{nom}1 : effectue un renvoi vers le numéro de la page qui
  correspond au repère \emph{nom}. Pour le reste, il fonctionne exactement de la
  même manière que \verb1\ref1.
\end{itemize}
                                % \label \ref \pageref

Nous verrons ultérieurement que le mode Emacs \emph{refTeX}
(Chapitre~\ref{cha:reftex}, p.\pageref{cha:reftex}) peut faciliter le
travail de gestion des références.



 \section{Création de documents PDF}

 Il existe (au-moins) deux méthodes pour obtenir un document PDF à
 partir d'un fichier source \LaTeX:

 \begin{description}
 \item[classique] \LaTeX\ $\Rightarrow$ DVI $\Rightarrow$ PS
   $\Rightarrow$ PDF
 %\item[dvipdfm] \LaTeX\ $\Rightarrow$ DVI $\Rightarrow$ PDF
 \item[pdflatex] \LaTeX\ $\Rightarrow$ PDF
 \end{description}

 Nous utiliserons la dernière méthode. Elle permet d'insérer des
 graphiques dans des formats assez fréquents sous Windows et Mac OS
 (JPEG, PNG, mais aussi PDF).

 La méthode classique peut être intéressante pour ceux qui utilisent
 l'extension \emph{PSTricks} qui fournit des outils intéressants pour
 la génération de graphes (mais on peut aussi utiliser
 \emph{tikz}). Elle nécessite cependant l'usage de graphiques au
 format \emph{EPS} (Embedded Postscript) (peu utilisés sous Windows,
 en tout cas peu connus des utilisateurs\footnote{Postscript est le
   format de prédilection utilisé sous Unix})


 Dans emacs, vous pouvez sélectionner la création automatique directe
 de fichiers PDF en sélectionnant dans le menu Command $\Rightarrow$
 TeXing Options, la case \emph{PDF Mode}.

 Pour indiquer à Emacs de toujours sélectionner cette option au
 démarrage, le plus simple est d'ajouter, dans le fichier
 \verb1.emacs1 situé dans votre répertoire personnel, la ligne :
 \begin{quote}
   \verb1(add-hook 'LaTeX-mode-hook 'TeX-PDF-mode)1
 \end{quote}
 et de redémarrer emacs.

Il est aussi possible de configurer emacs depuis les menus :

\begin{quote}
  Options $\Rightarrow$ Customize Emacs $\Rightarrow$ Browse
  Customisation Groups
\end{quote}

et aller ensuite dans :

\begin{quote}
  Emacs $\Rightarrow$ Wp $\Rightarrow$ Tex $\Rightarrow$ AUCTeX
  $\Rightarrow$ Tex Pdf Mode
\end{quote}

Cliquer sur \og Toggle \fg pour faire passer cette option à ON (cette
opération est inutile si vous passez par la modification du fichier
\verb1.emacs1).
