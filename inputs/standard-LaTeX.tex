


%%%%%%%%%%%%%%%%%%%%%%%%%%%%%%%%%%%%%%%%%%%%%%%%%%%%%%%%%%%%%%%%%%%%%%%%%%%

%%  STANDARD LaTeX

%%%%%%%%%%%%%%%%%%%%%%%%%%%%%%%%%%%%%%%%%%%%%%%%%%%%%%%%%%%%%%%%%%%%%%%%%%%







\section{\LaTeX{} et les caractères accentués}

Créez un nouveau document en utilisant des caractères accentués dans le texte :

\begin{exemple}[H] %\footnotesize
  %\footnotesize
%\caption{Usage des caractères accentués avec \LaTeX.}
\begin{verbatim}
\documentclass[a4paper]{article}
\begin{document}
    Rédigeons un document \LaTeX.
\end{document}
\end{verbatim}
\end{exemple}

Que remarquez-vous dans le fichier DVI ?

Afin d'utiliser les caractères accentués, deux solutions s'offrent à
nous :

\begin{enumerate}
\item Utiliser \verb1\'e1 pour é, \verb1\`e1 pour è, \verb1\^e1 pour
  ê, \verb1\"e1 pour ë\ldots On fait de même pour n'importe quelle
  autre lettre (on peut même ainsi obtenir \"p, \^z\ldots des
  majuscules accentuées\footnote{Notez que les règles de typographie
    française imposent l'accentuation des majuscules} : \'E, \`A,
  \ldots).
\item Ajouter, dans le préambule du document l'appel à
  l'\emph{extension} \og inputenc \fg\ (La méthode précédente reste
  valable) : \verb+\usepackage[option]{inputenc}+..
\end{enumerate}

On obtient alors :
\begin{exemple}[H] %\footnotesize
  %\footnotesize
  \caption{Usage des caractères accentués avec \LaTeX (encodage
    iso-8859-1).}
\begin{verbatim}
\documentclass[a4paper]{article}
\usepackage[utf8]{inputenc}

\begin{document}
    Rédigeons un document \LaTeX.
\end{document}
\end{verbatim}
\end{exemple}
% options à inputenc : latin1, cp437, cp850, applemac notamment

En fonction de votre système, il se peut qu'il soit configuré pour
écrire des documents dans l'encodage unicode (utf8). Pour ce qui
concerne emacs, lorsque vous créez un fichier pour la première fois,
vous pouvez voir quel encodage il choisit en regardant en bas à gauche
de la barre de statut. Si vous observez \verb?-1:?, emacs encode votre
texte en iso-8859-1 (latin1 pour \LaTeX). Si vous observez \verb?-u:?,
emacs l'encode par défaut en unicode (utf8). Vous devrez alors
remplacer l'appel à inputenc produit précédemment par :
\begin{exemple}[H] %\footnotesize
  %\footnotesize
  \caption{Usage des caractères accentués avec \LaTeX (encodage
    unicode).}
\begin{verbatim}
\usepackage[utf8]{inputenc}
\end{verbatim}
\end{exemple}

D'autres encodages pourraient être utilisés par votre système. Référez
vous à sa documentation.


\section{L'instruction \emph{usepackage}}

\vfill
\begin{itemize}
\item On distingue dans un document \LaTeX, d'une part le corps du
  document (corresondant à tout ce qui est situé entre les deux
  extrémités de l'environnement \emph{document};
\item d'autre part l'en-tête du document (entre \verb1\documentclass1
  et \verb1\begin{document}1);
  \item Toutes les instructions \verb1\usepackage{}1 doivent être
    utilisée dans l'en-tête du document uniquement, jamais dans le
    corps du document;
\end{itemize}
\vfill

\section{Liste des accents possibles}


\begin{tabbing}
\verb1\`{o}1 \hspace{1cm}\=\a`{o} \hspace{1cm}\=\\
\verb1\'{o}1 \>\a'{o} \\
\verb1\^{o}1 \>\^{o} \\
\verb1\"{o}1 \>\"{o} \\
\verb1\H{o}1 \>\H{o} \> $\rightarrow$ \og Accent Hongrois \fg \\
\verb1\~{o}1 \>\~{o} \\
\verb1\c{c}1 \>\c{c} \\
\verb1\={o}1 \>\a={o} \\
\verb1\b{o}1 \>\b{o} \\
\verb1\.{o}1 \>\.{o} \\
\verb1\d{o}1 \>\d{o} \\
\verb1\u{o}1 \>\u{o} \\
\verb1\v{o}1 \>\v{o} \\
\verb1\t{oo}1 \>\t{oo} \> $\rightarrow$ de l'anglais \emph{tie}, \og lien \fg 
\end{tabbing}
       

\begin{itemize}
\item Notez que les accolades autour de la lettre à accentuer ne sont
  obligatoires que lorsque le type d'accent est indiqué par une lettre
  (H, c, b, d, u, v, t). Ainsi, comme indiqué page précédente, il est
  suffisant d'entrer \verb1\'o1 pour obtenir \'o.
\item Si l'on souhaite introduire un accent sur une lettre déjà dotée
  d'un point (i ou j), il existe une méthode qui permet de supprimer
  ce point : \verb1\i1 ou \verb1\j1 donnent respectivement \i\ et \j.
\item Ainsi, pour écrire un i avec un tilde, il suffit d'entrer
  \verb1\~\i1 et on obtient \~\i. Sinon, on obtient \~i.
\item Par contre, si vous utilisez l'extension \og inputenc \fg, vous
  pouvez entrer directement un \emph{î} ou un \emph{ï}. Le point sur
  le i n'apparaîtra pas. Ceci est particulièrement utile pour le
  français.
\item Vous pouvez mélanger les deux méthodes sans aucun problème.
\end{itemize}

\newpage
\vfill
\begin{description}
\item[Exercice :] Entraînez-vous à écrire des lettres avec des accents
  en utilisant les deux méthodes ;
  \begin{itemize}
  \item La méthode \og inputenc \fg\ pour ceux qui sont disponibles
    sur votre clavier ;
  \item Le méthode avec \emph{déclaration préalable} de l'accent pour
    ceux qui ne sont pas disponibles au clavier. Essayez notamment les
    accents sur les majuscules et sur les lettres qui correspondent à
    des consonnes.
  \end{itemize}
\end{description}
\vfill





\section{Traitement des espaces}

\subsection{\LaTeX{} et les espaces}

Dans un fichier \LaTeX{}, une série d'espaces est interprétée comme un
seul espace. Par conséquent, vous pouvez taper deux, trois, mille
espaces à la suite, ça ne changera pas l'espacement entre les mots.
\begin{example}
Saisir un ou           10 
espaces entre les mots n'a 
aucune importance.
\end{example}


\subsection{Sauts de lignes}

Pour \LaTeX, un saut de ligne ne signifie pas grand chose. Vous pouvez
ainsi écrire :

\begin{footnotesize}
\begin{verbatim}
Je peux écrire mes phrases sans faire
attention
aux
sauts de
lignes entre les mots et j'obtiens quand même une mise en page parfaite.
etc. etc. etc.
etc. etc. etc.
etc. etc. etc.
etc. etc. etc.
etc. etc. etc.
etc. etc. etc.
etc. etc. etc.
etc. etc. etc.
etc. etc. etc.
\end{verbatim}
\end{footnotesize}

Ce qui donne :

\begin{boxedminipage}{.5\textwidth}
Je peux écrire mes phrases sans faire
attention
aux
sauts de
lignes entre les mots et j'obtiens quand même une mise en page parfaite.
etc. etc. etc.
etc. etc. etc.
etc. etc. etc.
etc. etc. etc.
etc. etc. etc.
etc. etc. etc.
etc. etc. etc.
etc. etc. etc.
etc. etc. etc.
\end{boxedminipage}


\subsection{Changements de paragraphe}

Par contre, pour changer de paragraphe, il faut introduire une ligne
vide.

\begin{footnotesize}
\begin{verbatim}
Ceci est un premier paragraphe qui
contient un nombre de mots suffisant pour obtenir un paragraphe de plusieurs
lignes et il faut sauter une
ligne pour changer de paragraphe.

Voici mon 
second paragraphe. Et j'ajoute quelques mots pour obtenir plusieurs
lignes.
\end{verbatim}
\end{footnotesize}

Ce qui donne :

\begin{boxedminipage}{.5\textwidth}
Ceci est un premier paragraphe qui
contient un nombre de mots suffisant pour obtenir un paragraphe de plusieurs
lignes et il faut sauter une
ligne pour changer de paragraphe.

Voici mon 
second paragraphe. Et j'ajoute quelques mots pour obtenir plusieurs
lignes.
\end{boxedminipage}

Pour reformater vos paragraphes afin qu'ils soient plus lisibles dans
emacs, taper ALT+Q lorsque le curseur est placé sur l'une des lignes
du paragraphe à reformater.


\section{Caractères réservés}

Les caractères suivants ne peuvent pas être entrés tels quels au
clavier car ils ont une signification particulière pour le formateur
\LaTeX :

\begin{flushleft}
\begin{tabbing}
\#\hspace{1cm}\=
\&\hspace{1cm}\=
\%\hspace{1cm}\=
\$\hspace{1cm}\=
\{\hspace{1cm}\=
\}\hspace{1cm}\=
\_\hspace{1cm}\=
\~{} \hspace{1cm}\=
\^{}\hspace{1cm}\=
\textbackslash
\end{tabbing}
\end{flushleft}

Pour obtenir ces caractères dans le fichier mis en page, il faut pour la plupart
d'entre eux les faire précéder d'une barre oblique inversée (backslash, ou
caractère d'échappement) :

\begin{flushleft}
\begin{tabbing}
\#\hspace{1cm}\=
\&\hspace{1cm}\=
\%\hspace{1cm}\=
\$\hspace{1cm}\=
\{\hspace{1cm}\=
\}\hspace{1cm}\=
\_\hspace{1cm}
\\
\verb1\#1\>
\verb1\&1\>
\verb1\%1\>
\verb1\$1\>
\verb1\{1\>
\verb1\}1\>
\verb1\_1
\end{tabbing}
\end{flushleft}

Les trois derniers sont un peut plus compliqués :
\begin{itemize}
\item \~{} doit être entré \verb1\~{}1 (car \verb1\~1 sans les accolades
  mettrait un tilde sur la lettre suivante). Ceci revient à mettre un tilde au
  dessus de \emph{rien}.;
\item \^{} doit être entré \verb1\^{}1 (pour la même raison) ;
\item \textbackslash doit être entré \verb1\textbackslash1 car pour \LaTeX, la
  séquence \verb1\\1 indique une coupure de ligne (un saut de ligne).
\end{itemize}



\section{Les composants d'un document \LaTeX}

\begin{itemize}
\item Il existe deux grandes catégories de composants auxquels fait
  appel un document \LaTeX.
  \begin{itemize}
  \item La \emph{classe de document} : c'est un composant obligatoire
    du document. Il va apporter à \LaTeX\ toutes les informations dont
    il a besoin pour assurer la mise en forme du texte. Il fournit
    également un certain nombre de déclarations qui ne sont pas
    nécessairement disponibles dans l'interpréteur \LaTeX.
  \item Les \emph{extensions} : ce sont des composants optionnels
    auxquels fait appel le document pour sa mise en forme mais aussi
    pour faciliter la rédaction (cf. l'extension \og inputenc \fg).
  \end{itemize}
\item Pour le moment, nous allons nous contenter de travailler avec le
  c\oe{}ur de \LaTeX. Nous n'utiliserons donc aucune extension à
  l'exception d'\og inputenc \fg\ et de \og babel \fg mais d'autres
  extensions viendront plus tard.
\end{itemize}





\section{Classes de documents}

\begin{itemize}
\item Tout fichier \LaTeX\ commence par une indication sur le type (la
  classe) de document rédigé (article, livre, lettre\ldots). C'est le
  rôle de la déclaration \verb1\documentclass1.
\item Chaque classe de document apporte :
  \begin{itemize}
  \item Une mise en page particulière;
  \item Des fonctionnalités spécifiques.
  \end{itemize}
\item Chaque classe de document accepte également un certain nombre
  d'options qui permettent d'influencer la mise en page.
\item Il est à noter que, par défaut, l'utilisateur n'a pas à se
  préoccuper de la mise en page de son document. \LaTeX\ procèdera
  automatiquement à une mise en forme de qualité (positionnement des
  parties du texte, choix des tailles et des types de polices\ldots)
  sans aucune intervention de la part de l'utilisateur qui n'a à se
  soucier que du contenu.

  La déclaration de la classe de document est de la forme :

  \begin{center}
    \verb1\documentclass[liste des options]{classe du document}1
  \end{center}

\end{itemize}

Les différentes options sont séparées par des virgules.

\section{La classe de document \og article \fg}

%% %\footnotesize
\begin{flushright}
{\Large \verb1\documentclass{article}1}
\end{flushright}

\begin{itemize}
\item Elle accepte entre autres les options suivantes :
  \begin{itemize}
  \item a4paper | letterpaper \hfill (format du papier)
  \item portrait | landscape \hfill (orientation)
  \item 12pt | 11pt | 10pt \hfill (taille de la police par défaut)
  \item oneside | twoside \hfill (impression recto-verso ou pas)
  \item final | draft \hfill (inclure les graphiques ou pas)
  \item notitlepage | titlepage \hfill (la page de titre doit-elle
    être séparée du reste du texte)
  \item onecolumn | twocolumn \hfill (nombre de colonnes de texte pour
    le formatage)
%  \item leqno, fleqn, openbib
  \end{itemize}
\item Si vous utilisez plusieurs options simultanément, il faut les
  séparer par une virgule.
\item Par exemple, pour imprimer un article sur deux colonnes en recto
  simple avec une police par défaut de 11pt sur du papier A4 sans
  inclure les graphiques, il faut indiquer :

  \verb=\documentclass[a4paper, 11pt,draft, oneside]{article}=
  
\item Notez qu'un certain nombre de ces options sont activées par
  défaut. Sur la plupart des installations, appeler
  \verb=\documentclass{article}= sans options revient à appeler :

  \verb=\documentclass[letterpaper,10pt,oneside,final,onecolumn]{article}=

  ou

  \verb=\documentclass[a4paper,10pt,oneside,final,onecolumn]{article}=

\end{itemize}

%% \verb=\documentclass[a4paper|letterpaper,portrait|landscape,12pt|10pt|11pt,oneside|twoside=
%% \verb=                        ,final|draft,notitlepage|titlepage,onecolumn|twocolumn=
%% \verb=                        ,leqno,fleqn,openbib]{article}=

%\newpage

\begin{itemize}
\item La classe \og article \fg\ permet d'utiliser 5 niveaux de
  section :
  \begin{itemize}
  \item \verb1\part1
  \item \verb1\section1
  \item \verb1\subsection1
  \item \verb1\subsubsection1
  \item \verb1\paragraph1
  \item \verb1\subparagraph1
  \end{itemize}
\item Elle fournit notamment 3 déclarations importantes qui doivent
  être positionnées \emph{avant} l'instruction \verb1\begin{document}1
    :
  \begin{itemize}
  \item \verb1\title1 \hfill (le titre de l'article : \verb1\title{mon titre}1)
  \item \verb1\author1 \hfill (le nom du (ou des) auteur(s) : \verb1\author{auteur-A \& auteur-B}1)
  \item \verb1\date1 \hfill (la date : essayez \verb1\date{\today}1 ou \verb1\date{}1)
  \end{itemize}
\item \verb1\maketitle1 inséré après \verb1\begin{document}1 met en page la page
    de titre.
\end{itemize}


\section{Autres classes de documents}

\begin{flushright}
  {\Large \verb1\documentclass{book}1} (pour des livres, thèses)\\
  {\Large \verb1\documentclass{report}1} (pour des livres, mémoires, thèses, rapports)\\
  {\Large \verb1\documentclass{letter}1} (pour des lettres)\\
  {\Large \verb1\documentclass{lettre}1} (pour des lettres aussi mais les possibilités sont plus étendues)\\
  {\Large \verb1\documentclass{a0poster}1} (pour des posters)\\
  {\Large \verb1\documentclass{prosper}1} (pour des présentations multimédia)\\
  {\Large \verb1\documentclass{beamer}1} (pour des présentations multimédia aussi)\\
  {\Large \verb1\documentclass{article}1}\\
  \ldots
\end{flushright}



\section{Les classes \emph{book} et \emph{report}}

Ces deux classes de documents (entre autres) vous donnent notamment
accès à un niveau supplémentaires de sectionnement :

\begin{itemize}
\item \verb1\chapter1
\end{itemize}

qui viennent s'ajouter à ceux que nous avons vus :

\begin{itemize}
\item \verb1\part1
\item \verb1\chapter1
\item \verb1\section1
\item \verb1\subsection1
\item \verb1\subsubsection1
\item \verb1\paragraph1
\item \verb1\subparagraph1
\end{itemize}


\section{Mise en pratique}

\vfill
\begin{description}
\item[Exercice :] Entraînez-vous à rédiger un document un peu plus
  long que ce que nous avons écrit jusqu'ici (1 à 2 pages) en
  structurant votre texte. Vous pouvez par exemple imaginer le plan
  d'un article scientifique ou journalistique, un récit de voyage, la
  description de l'histoire d'un roman ou d'un film\ldots
\end{description}
\begin{itemize}
\item Une fois votre plan rédigé sur papier (juste les titres),
  tapez-le dans Emacs en utilisant le marquage \LaTeX\
  (\verb1\section, \subsection,1\ldots).
  \begin{itemize}
  \item Profitez-en pour jouer avec les accents, les caractères
    réservés, les espaces entre les mots, le reformatage des
    paragraphes dans emacs\ldots
  \item Rédigez un texte qui fasse une à deux pages a4 après
    compilation (ce texte peut parfaitement n'avoir aucun sens) en
    introduisant un ou plusieurs paragraphes par section.
  \item Prenez votre temps pour vous habituer à l'interface d'Emacs et
    reprendre les points que nous avons vus ensemble à la séance
    précédente (raccourcis clavier notamment).
  \end{itemize}
\end{itemize}
\vfill
