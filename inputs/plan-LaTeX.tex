\section{Plan de la formation \LaTeX}

\begin{enumerate}
  
  %% 4h
\item Introduction
  \begin{enumerate}
  \item Présentation de \TeX\ \& \LaTeX
  \item Quelques sources d'information indispensables
  \item Comparaisons entre les différents systèmes disponibles
  \item Premiers pas avec \LaTeX
  \item Initiation à l'utilisation d'\emph{Emacs}
  \end{enumerate}
  
  %% 4h
\item Rédaction de documents avec \LaTeX\
  \begin{enumerate}
  \item \LaTeX\ et les caractères accentués
  \item \LaTeX, mise en page, typographie
  \item Les caractères spéciaux (caractères réservés)
  \item Les composants d'un document \LaTeX\ (classe de documents, extensions,
    options)
  \item La classe de documents \emph{article} : votre premier document avec
    \LaTeX
  \end{enumerate}

%\item Formats de fichiers à connaître (tex, dvi, ps, pdf)
%\item Création de documents imprimables~: dvips, ghostscript, pdflatex
  
% 6h
\item \LaTeX : Utilisation avancée
  \begin{enumerate}
  \item Comment créer un fichier PDF à partir d'un fichier \LaTeX
  \item Les flottants -- 1 : graphiques
    \begin{enumerate}
    \item Insérer une image
    \item Positionner automatiquement un graphique sur la page : la notion de
      \og flottant \fg
    \end{enumerate}
  \item Les flottants -- 2 : tableaux
    \begin{enumerate}
    \item Construire un tableau
    \item Les tableaux flottants
    \item Les tableaux avec saut de page
    \end{enumerate}
  \end{enumerate}


\item Les éléments importants lors de la rédaction d'un document

  \begin{enumerate}

  \item Les principaux environnements disponibles
    
  \item Renvois, Indexes, Tables de matières, Annexes
    
  \end{enumerate}

\item La gestion des références bibliographiques (\textsc{BIB}\TeX)
    
  \begin{enumerate}
  \item \LaTeX et \BiBTeX
  \item Créer un fichier \BiBTeX : une base de données bibliographiques
  \item Citer les références dans le texte : version standard + version natbib
  \item Compiler le document pour formater automatiquement les références
  \end{enumerate}


% \item \LaTeX et les autres formats de documents

%   \begin{enumerate}
%   \item DVI, PS, PDF
%   \item Exporter vers RTF, HTML
%   \item Convertir des documents Word vers \LaTeX
%   \item Exporter des graphiques depuis Excel au format \og image \fg
%   \end{enumerate}

\item Comment obtenir des mises en page personnalisées
    
  \begin{enumerate}
  \item les autres classes de documents
    \begin{enumerate}
    \item Classes standard
      \begin{itemize}
      \item book, report : pour les thèses
      \item letter : pour les\ldots\ lettres
      \item seminar, prosper, beamer : pour les transparents et présentations
        multimédia
      \item a0poster : pour les posters
      \item apa.cls : American Psychological Association (pour les soumissions
        d'articles)
      \item etc.
      \end{itemize}
    \end{enumerate}
  \item Trucs et astuces pour la personnalisation
    \begin{enumerate}
    \item des marges, de l'espacement entre les lignes
    \item des titres de section
    \item des en-têtes et pieds de page
    \item et divers points de personnalisation (polices, veuves et orphelines,
      indentation, espacement des paragraphes, notes)
    \item Créer son fichier d'extension personnel ou un gabarit standard
    \end{enumerate}
  \end{enumerate}
  
  %% \item Les polices de caractères
  
  \item L'extension \emph{babel} ou comment écrire dans des langues autres que
    l'anglais
  
  %%\item PSTricks~: un outil pour générer des graphiques et positionner les objets
  %%  dans un document (très utile pour les posters et les transparents).
  
  \item Problèmes divers
  
    \begin{enumerate}
    \item Correction orthographique avec Emacs et Ispell
    \item Babel et le francais
    \item Besoins spécifiques (Schémas, Organigrammes --picture, eepic--,
      \'Equations (ams)
    \item Problèmes posés par les logiciels ou outils spécifiques et leur
      utilisation avec \LaTeX (tableaux ou graphiques sous Excel -- ou
      importation de données d'autres logiciels\ldots, Bases de données
      bibliographiques et exportation au format \BiBTeX).
    \end{enumerate}
    
  \item Outils spécifiques pour la linguistique :
    
    \begin{enumerate}
    \item TIPA : les polices phonétiques 
    \item AVM : les matrices de traits (cf.  phonologie déclarative, HPSG) 
    \item qtree et treedvips : les arbres (syntaxiques notamment).
    \end{enumerate}
  
\end{enumerate}


% \section{Autres points prévus}

% \vfill
% \begin{itemize}
% % \item Nous consacrerons également quelques heures à des exercices de recherche
% %   d'information dans la FAQ ou dans la documentation de \LaTeX\ afin de résoudre
% %   des problèmes précis.
% % \item Nous aborderons également le problème de la conversion de documents Word
% %   vers le format \LaTeX\ (et inversement).
% \item Je pourrai vous distribuer un CD de la distribution TeXLive 2003. Et nous
%   pourrons aussi voir rapidement quelles autres versions du formateur \LaTeX\ 
%   sont disponibles.
% \end{itemize}
% \vfill

