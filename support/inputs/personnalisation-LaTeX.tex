
\section{Le contrôle des tailles avec \LaTeX}

Dans les pages suivantes, un certain nombre de commandes contrôlent
des dimensions qui prises en compte par \LaTeX\ lors du formatage. Les
dimensions peuvent être exprimées dans différentes unités, parmi
lesquelles :

Les quatre premières unités (cm, mm, in, pt) sont des unités absolues
(leur valeur reste la même quel que soit le contexte dans lequel on se
trouve). Les deux dernières sont des unités dites relatives (leur
valeur réelle dépend du contexte, c'est à dire de la taille de la
police en cours).


\begin{description}
\item[cm] centimètres;
\item[mm] millimètres;
\item[in] pouces (inches);
\item[pt] points (dimension informatique);
\item[ex] taille verticale d'un \emph{x} (permet d'indiquer une taille
  relative à la police utilisée);
\item[em] taille horizontale d'un \emph{m} (permet également
  d'indiquer une taille relative à la police utilisée);
\end{description}

\section{Double interligne}

Dans la plupart des documents universitaires (mais aussi parfois lors
de la soumission d'articles dans des revues scientifiques), il vous
est demandé de fournir des textes avec une interligne double. Il
existe une extension destinée spécifiquement à cet usage :

\begin{verbatim}
\usepackage{setspace}
\doublespacing

ou

\onehalfspacing

\end{verbatim}

Comme d'habitude, l'appel de l'extension se fait dans l'en-tête du
document.

La commande \verb1\doublespacing1 (ou \verb1\onehalfspacing1) est
aussi introduite dans l'en-tête; elle est valable pour l'ensemble du
document.

A priori, vous \emph{ne devez pas} changer ce paramètre en cours de
document. Si toutefois vous tenez absolument à le faire, l'extension
fournit aussi des environnement permettant de changer temporairement
le réglage.


\section{Marges}

L'extension \emph{geometry} permet de contrôler précisément la taille
des marges du document (entre autres). Son utilisation est très simple
:

\begin{verbatim}
\usepackage[left=5cm,right=4cm,bottom=2.5cm,top=2.5cm]{geometry}
\end{verbatim}






\section{Formatage de la page de titre}

Pour formater votre page de titre, vous pouvez, à la place de
l'instruction \verb1\maketitle1 vue au début du cours, utiliser un
code similaire à ce qui suit et l'adapter à votre convenance :


\begin{footnotesize}
\begin{verbatim}
\begin{titlepage}
  \singlespacing
  \sffamily
  \begin{center}
    {\bfseries\large Université de Nantes}\\
    {UFR Lettres et Langages}\\
    {Année Universitaire 2005--2006}\\
    \vfill

    {\Large\bfseries Titre du Mémoire}

    \vspace{2ex} {\large \'Eventuellement un sous-titre}
     
    \vspace{8ex}
    Prénom Nom\\
    Date

    \begin{flushright}
      \vfill
      \begin{tabular}{lr}
        Membres du Jury : & \\
        & Nom, Prénom (directeur du mémoire)\\
        & Nom, prénom\\
      \end{tabular}
    \end{flushright}
    \vspace{8ex}
    
    { Master 2 mention \og Langues et Langages \fg\\
      Spécialité \og Sciences du Langage \fg\\
    Parcours \og Informatisation des Langues \fg\\}
  \end{center}
\end{titlepage}
\end{verbatim}
\end{footnotesize}




\section{En-têtes et Pieds de pages}

Pour contrôler les en-têtes et les pieds de page, on utilise
l'extension \emph{fancyhdr} (pour \emph{fancy headers}). On déclare
que le style de page utilisé sera le style \emph{fancy} (celui produit
par cette extension) et on définit la nature des différentes parties
de ces éléments.

On définit ci-dessous les caractéristiques de l'en-tête droite et
gauche et celles du pied de page au centre (dans lequel on va placer
les numéros de page).

On met ensuite la largeur des lignes de séparation entre l'en-tête et
le texte (et respectivement le pied de page) à 0 points (pas de
lignes).

Les commandes \verb1\rightmark1 et \verb1\leftmark1 permettent de
faire varier le texte de manière automatique (en fonction des classes
de documents, les en-têtes gauche et droit contiendront par exemple
les noms des auteurs, le titre du document, le titre du chapitre en
cours\ldots

\begin{verbatim}
\usepackage{fancyhdr}
\pagestyle{fancy}

\fancyhf{} % ici, on vide les en-têtes et pieds de page avant de les définir

\fancyhead[R]{\small \sffamily En-tête à droite} (on peut par exemple mettre \rightmark)
\fancyhead[L]{\small \sffamily En-tête à gauche} (on peut par exemple mettre \leftmark)
\fancyfoot[C]{\small \sffamily \thepage}
\renewcommand{\headrulewidth}{0pt}
\renewcommand{\footrulewidth}{0pt}
\end{verbatim}


\section{Personnalisation des titres de section (mode de numérotation)}

On contrôle le mode de numérotation en redéfinissant les commandes
suivantes (qui sont utilisées automatiquement par \LaTeX\ lors de la
compilation). Il convient, au préalable, de \emph{relever} la
profondeur de section au-delà de laquelle s'interrompt la numérotation
des sections.

{\small
\begin{verbatim}
\setcounter{secnumdepth}{7}

\def\thechapter{\Roman{chapter}}
\def\thesection{\Alph{section}}
\def\thesubsection{\Roman{subsection}}
\def\thesubsubsection{\arabic{subsubsection}}
\def\theparagraph{\roman{paragraph}}
\def\thesubparagraph{\alph{subparagraph}}
\end{verbatim}
}

\section{Personnalisation des titres de section (changement de mise en forme simple)}

On fait appel à l'extension \emph{sectsty} et on redéfinit chaque
niveau de section (alignement du texte, taille de la police, graisse,
type de police, éventuellement suppression de la numérotation des
pages, insertion d'un saut de page préalable).

{\small
\begin{verbatim}
\usepackage{sectsty}

%\partfont{\centering\Huge\bfseries\sffamily\thispagestyle{empty}}
%\chapterfont{\Large\bfseries\sffamily}
\sectionfont{\newpage\raggedleft\large\bfseries\sffamily}
\subsectionfont{\large\mdseries\sffamily}
\subsubsectionfont{\normalsize\mdseries\sffamily}
\end{verbatim}
}


\section{Personnalisation des titres de section (changement de mise en forme plus complexe)}

On fait appel à l'extension \emph{titlesec} (et ici on ajoute un
recours à l'extension \emph{scalefont} qui permet de gérer plus
finement les tailles de police). Le format des commandes est le suivant :

{\small
\begin{verbatim}
\titleformat{niveau}[formatage du paragraph]{formatage du titre}{formatage du numéro}...
                          ...{espacement entre numéro et titre}{code à insérer après le titre}
\end{verbatim}
}

{\scriptsize
\begin{verbatim}
\usepackage{scalefnt}
\usepackage{titlesec}

\titleformat{\part}[display]{\raggedleft\Huge\scalefont{1.3}\sffamily\bfseries\thispagestyle{empty}}{\partname}{0cm}{}
\titleformat{\chapter}[display]{\raggedleft\Huge\sffamily\bfseries}{\chaptertitlename\ \thechapter}{0cm}{}
\titleformat{\section}{\Large\sffamily\bfseries}{\thesection)\quad}{0cm}{}
\titleformat{\subsection}{\Large\sffamily}{\thesubsection --\quad}{0cm}{}
\titleformat{\subsubsection}{\large\sffamily\itshape\raggedright}{\thesubsubsection.\quad}{0cm}{}
\titleformat{\paragraph}{\normalsize\sffamily\itshape\raggedright}{\theparagraph\quad}{0cm}{}
\titleformat{\subparagraph}{\normalsize\sffamily\itshape\raggedright}{\thesubparagraph\quad}{0cm}{}
\end{verbatim}
}


\section{Modification des paragraphes}

On peut modifier très simplement l'indentation par défaut (espace en
début de ligne lors des changements de paragraphe) et l'espacement
vertical entre paragraphes :

\begin{verbatim}
\setlength{\parindent}{0cm}
\setlength{\parskip}{1em}
\end{verbatim}



\section{Et maintenant ?}

Si vous avez réussi à tenir jusqu'au bout, alors vous êtes sur la
bonne voie\ldots mais votre apprentissage est loin d'être fini. Il
existe très certainement un millier de choses que vous souhaiteriez
pouvoir faire sans encore savoir comment vous pourriez vous y prendre
avec \LaTeX. Il vous faudra faire preuve de beaucoup de patience et
d'un travail régulier, lire la documentation qui fourmille sur
internet, vous procurer un livre (éventuellement en format
électronique, cf. les références données p.~\pageref{sec:freebooks}),
rechercher des extensions qui pourraient répondre à vos besoins en
consultant le \emph{Comprehensive Tex Archive Network}
(p.~\pageref{sec:websites}).

N'hésitez pas aussi à rechercher la réponse à vos problèmes dans les
foires aux question (p.~\pageref{sec:faq}) et à consulter
régulièrement les \emph{newsgroups} (p.~\pageref{sec:newsgroups}).

\vfill

\hfill Bon courage\ldots et bonne route avec \LaTeX !

\vfill
