


\section{\LaTeX\ en linguistique}

Un certain nombre des besoins spécifiques à la linguistique peuvent
parfois être comblés par des outils généraux. Ainsi, la production de
schémas et autres graphiques peut se faire avec des outils comme
\emph{tikz} ou \emph{pstricks}.

D'autres outils sont vraiment spécifiques au linguiste, parmi eux
--évidemment-- la transcription phonétique est un outil indispensable
qui est fourni par l'extension \emph{TIPA} :

\bibentry{tipadoc}.

Pour une liste plus complète d'outils utiles en linguistique,
consultez le site internet \emph{latex4ling} (\LaTeX\ for linguists,
\url{http://www.essex.ac.uk/linguistics/clmt/latex4ling/}) qui
présente une liste très intéressantes d'extensions \LaTeX\ qui peuvent
s'avérer cruciales dans le travail linguistique. Ces différentes
extensions sont classées par \og thème \fg : Matrices (AVM), arbres,
représentations autosegmentales, grammaires catégorielles,
gloses\ldots


\section{TIPA}

Pour procéder à des transcriptions phonétiques, on utilise l'extension
\emph{TIPA} en l'appelant dans le préambule du document :

\begin{verbatim}
\usepackage{tipa}
\end{verbatim}

Puis, chaque fois qu'on souhaite produire une transcription
phonétique, on utilise l'instruction :

\begin{verbatim}
 \textipa{}
\end{verbatim}

Par exemple :

\begin{verbatim}
[\textipa{lepaKtisip\~A\~ObOjkotelaKesEpsj\~Omes\o kietepKez\~Ap\oe vdiKkilz\~Obj\~Efe}]
\end{verbatim}

Produit : [\textipa{lepaKtisip\~A\~ObOjkotelaKesEpsj\~Omes\o kietepKez\~Ap\oe vdiKkilz\~Obj\~Efe}]

Il vous faudra apprendre au moins les rudiments de la correspondance
entre les caractères phonétiques affichés et ceux que vous devez
entrer dans le fichier \LaTeX. Dans d'assez nombreux cas, il y a une
correspondance terme à terme avec assez souvent une proximité
graphique entre le caractère à entrer et celui qui apparaîtra. Pour
certains caractères cependant il faudra entrer des caractères (ou
suites de caractères) moins faciles à mémoriser. Un entraînement court
mais régulier vous permettra de maîtriser très rapidement ce
système. Lisez la documentation (\bibentry{tipadoc}) pour comprendre
comment on utilise TIPA.



\section{Quelques exemples simples de caractères phonétiques produits avec \emph{TIPA}}

\begin{center}
  \begin{tabular}{|l|cc|cc|cc|cc|cc|cc|cc|cc|cc|cc|cc|}
    \hline & 
    \multicolumn{2}{|c|}{\footnotesize{Bilabial}} &					% Bilabial
    \multicolumn{2}{|c|}{\footnotesize{Lab. dent.}} & 			% Labiodental
    \multicolumn{2}{|c|}{\footnotesize{Dental}} & 					% Dental
    \multicolumn{2}{|c|}{\footnotesize{Alveolar}} & 				% Alveolar
    \multicolumn{2}{|c|}{\footnotesize{P-alveo.}} & 		% Post-alveolar
    \multicolumn{2}{|c|}{\footnotesize{Retroflex}} & 				% Retroflex
    \multicolumn{2}{|c|}{\footnotesize{Palatal}} & 					% Palatal
    \multicolumn{2}{|c|}{\footnotesize{Velar}} & 					% Velar
    \multicolumn{2}{|c|}{\footnotesize{Uvular}} & 					% Uvular
    \multicolumn{2}{|c|}{\footnotesize{Pharyng.}} & 			% Pharyngeal
    \multicolumn{2}{|c|}{\footnotesize{Glottal}}  \\					% Glottal
    
    \hline Plosive &  							% Plosive
    p & b &													% Bilabial
    &&														% Labiodental
    \multicolumn{3}{|r}{t}&							% Dental
    \multicolumn{3}{l|}{d}&							% Alveolar
    % Post-alveolar
    \ipa{\:t} & \ipa{\:d}&									% Retroflex
    c & \textbardotlessj &														% Palatal
    k & g &													% Velar
    q & \ipa{\;G} &										% Uvular
    & \BlankCell        &								% Pharyngeal
    \ipa{P}& \BlankCell         \\								% Glottal
    
    \hline Nasal & 							% Nasal
    & m &													% Bilabial
    & \ipa{M} &											% Labiodental
    \multicolumn{3}{|r}{}&								% Dental
    \multicolumn{3}{l|}{n}&							% Alveolar
    % Post-alveolar
    & \ipa{\:n} &														% Retroflex
    & \textltailn &														% Palatal
    & \ipa{N} &														% Velar
    & \ipa{\;N} &														% Uvular
    \BlankCell        & \BlankCell        &		% Pharyngeal
    \BlankCell        & \BlankCell         \\		% Glottal
    
    \hline Trill &  								% Trill
    & \ipa{\;B}&											% Bilabial
    & &														% Labiodental
    \multicolumn{3}{|r}{}&								% Dental
    \multicolumn{3}{l|}{r}&								% Alveolar
    % Post-alveolar
    & &														% Retroflex
    & &														% Palatal
    \BlankCell        & \BlankCell        &		% Velar
    & \ipa{\;R}&											% Uvular
    & &														% Pharyngeal
    \BlankCell        & \BlankCell         \\		% Glottal
    
    \hline Tap/Flap &  						% Tap /Flap
    & &													% Bilabial
    & &														% Labiodental
    \multicolumn{3}{|r}{} &					% Dental
    \multicolumn{3}{l|}{\ipa{R}} &					% Alveolar
    % Post-alveolar
    & \ipa{\:r} &														% Retroflex
    & &														% Palatal
    \BlankCell        & \BlankCell        &		% Velar
    & &														% Uvular
    & &														% Pharyngeal
    \BlankCell        & \BlankCell         \\		% Glottal
    
    \hline Fricative & 						% Fricative
    \ipa{F} & \ipa{B} &									% Bilabial
    f & v &													% Labiodental
    \ipa{T} & \ipa{D} &									% Dental
    s & z &													% Alveolar
    \ipa{S} & \ipa{Z} &									% Post-alveolar
    \ipa{\:s} & \ipa{\:z} &								% Retroflex
    \ipa{\c{c}} & \ipa{J} &								% Palatal
    x & \ipa{G} &											% Velar
    \ipa{X} & \ipa{K} &									% Uvular
    \textcrh & \ipa{Q} &								% Pharyngeal
    h & \texthth \\										% Glottal
    
    \hline Lat. Fric. & 					% Lat. Fricative
    \BlankCell        & \BlankCell        &		% Bilabial
    \BlankCell        & \BlankCell        &		% Labiodental
    \multicolumn{3}{|r}{\textbeltl} &				% Dental
    \multicolumn{3}{l|}{\textlyoghlig} &			% Alveolar
    % Post-alveolar
    & &														% Retroflex
    & &														% Palatal
    & &														% Velar
    & &														% Uvular
    \BlankCell        & \BlankCell        			% Pharyngeal
    & \BlankCell        & \BlankCell         \\   % Glottal
    
    \hline Approx & 							% Approx.
    & &														% Bilabial
    & \ipa{V} &											% Labiodental
    \multicolumn{3}{|r}{}&								% Dental
    \multicolumn{3}{l|}{\ipa{\*r}} &					% Alveolar
    % Post-alveolar
    & \ipa{\:R} &											% Retroflex
    & j &														% Palatal
    & \textturnmrleg &									% Velar
    & &														% Uvular
    & &														% Pharyngeal
    \BlankCell        & \BlankCell         \\		% Glottal
    
    \hline Lat. appr. & 					% Lat. Approx
    \BlankCell        & \BlankCell        &		% Bilabial
    \BlankCell        & \BlankCell        &		% Labiodental
    \multicolumn{3}{|r}{}&								% Dental
    \multicolumn{3}{l|}{l}&								% Alveolar
    % Post-alveolar
    & \ipa{\:L} &											% Retroflex
    & \ipa{L} &												% Palatal
    & \ipa{\;L} &											% Velar
    & &														% Uvular
    \BlankCell        & \BlankCell        &		% Pharyngeal
    \BlankCell        & \BlankCell         \\		% Glottal
    \hline
  \end{tabular}
\end{center}


\begin{center}
\Large
  \begin{vowel}
    % \putcvowel[l]{i}{1}
    \putvowel[l]{i}{0pt}{0pt}
    \putcvowel[r]{y}{1}
    \putcvowel[l]{e}{2}
    \putcvowel[r]{\o}{2}
    \putcvowel[l]{\textepsilon}{3}
    \putcvowel[r]{\oe}{3}
    \putcvowel[l]{a}{4}
    \putcvowel[r]{\textscoelig}{4}
    \putcvowel[l]{\textscripta}{5}
    \putcvowel[r]{\textturnscripta}{5}
    \putcvowel[l]{\textturnv}{6}
    \putcvowel[r]{\textopeno}{6}
    \putcvowel[l]{\textramshorns}{7}
    \putcvowel[r]{o}{7}
    \putcvowel[l]{\textturnm}{8}
    \putcvowel[r]{u}{8}
    \putcvowel[l]{\textbari}{9}
    \putcvowel[r]{\textbaru}{9}
    \putcvowel[l]{\textreve}{10}
    \putcvowel[r]{\textbaro}{10}
    \putcvowel{\textschwa}{11}
    \putcvowel[l]{\textrevepsilon}{12}
    \putcvowel[r]{\textcloserevepsilon}{12}
    \putcvowel{\textsci\ \textscy}{13}
    \putcvowel{\textupsilon}{14}
    \putcvowel{\textturna}{15}
    \putcvowel{\ae}{16}
  \end{vowel}
\end{center} 
