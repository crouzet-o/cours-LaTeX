


\section{Les bases de l'indexation}


\verb1\usepackage{index}1

\verb1\proofmodetrue1

\verb1\newindex{default}{idx}{ind}{Index des notions}1 ou \verb1\makeindex1

\verb1\newindex{aut}{adx}{and}{Index des auteurs}1

\begin{verbatim}
  latex file.tex
  makeindex (= makeindex -o file.ind file.idx)
  makeindex -o file.and file.adx
  latex file.tex

  latex file.tex && makeindex file.idx && latex file.tex
\end{verbatim}

\verb1\makeindex1 (dans le préambule du document)

mot à indexer\verb1\index{mot à indexer}1

ou

\verb1\index{mot à indexer}1mot à indexer (pas d'espace au cas où un saut de
page venait séparer le mot de son indexation.)

ou

\verb1\index*{mot à index}1



\verb1\printindex1 à l'endroit où l'on souhaite faire apparaître l'index.







%%% Local Variables: 
%%% mode: latex
%%% TeX-master: "recueil-initiation-LaTeX"
%%% End: 
