\section{Créer un tableau avec les outils standards}
\label{tabular}

\vfill

\begin{boxedminipage}{\textwidth}
\begin{verbatim}
\begin{tabular}{spécification des colonnes}
donnée & donnée & donnée & ... & donnée \\
donnée & donnée & donnée & ... & donnée \\
...
\end{tabular}
\end{verbatim}
\end{boxedminipage}

\begin{itemize}
\item Instructions utiles
  \begin{itemize}
  \item \verb1\hline1 : trace une ligne horizontale sur toute la largeur du
    tableau;
  \item \verb1\cline{N-M}1 : trace une ligne horizontale des colonnes N à M;
  \item \verb1\multicolumn{nbcols}{format}{Texte étendu sur plusieurs colonnes}1
  \item la déclaration des colonnes se fait avec les lettres \emph{c,
      l, r} (center, left, right). On indique autant de lettres que
    l'on veut de colonnes, chaque lettre indique l'alignement du texte
    dans la colonne;
  \item Les \verb1&1 servent à sépararer les colonnes;
  \item On indique la fin de ligne par \verb1\\1.
  \end{itemize}
\end{itemize}

%%Nous allons voir quelques exemples de l'utilisation de \emph{tabular} :
\vfill

\section{Exercices}

\begin{itemize}
\item Construisez, sur papier, un tableau très simple (3 colonnes, 5
  lignes) et reproduisez le avec \LaTeX\ en traçant toutes les lignes
  verticales et horizontales;
\item Reproduisez ce même tableau en supprimant toutes les lignes;
\item Reproduisez pour finir ce même tableau en n'introduisant que les
  lignes horizontales;
\item Reproduisez ensuite le tableau présenté ci-dessous en utilisant
  les instructions nécessaires.
\end{itemize}

\begin{center}
  \begin{tabular}{|l|cr|l|} \hline
    Alignement à gauche & Données centrées & et à droite & Largeur imposée\\ \cline{2-3}
    Donnée & \multicolumn{2}{|c|}{Donnée} & \\ \hline
    1.10 & 2.20 & 3.20 & 4.69 \\
    1.10 & 2.20 & 3.20 & 4.69 \\
    1.10 & 2.20 & 3.20 & 4.69 \\ \hline
  \end{tabular}
\end{center}


\section{L'environnement \emph{tabular}: réponse à la question
  précédente}

\begin{boxedminipage}{\textwidth}
\footnotesize
\begin{verbatim}
\begin{center}
  \begin{tabular}{|l|cr|l|} \hline
    Alignement à gauche & Données centrées & et à droite & Largeur imposée\\ \cline{2-3}
    Donnée & \multicolumn{2}{|c|}{Donnée} & \\ \hline
    1.10 & 2.20 & 3.20 & 4.69 \\
    1.10 & 2.20 & 3.20 & 4.69 \\
    1.10 & 2.20 & 3.20 & 4.69 \\ \hline
  \end{tabular}
\end{center}
\end{verbatim}
\end{boxedminipage}

\begin{center}
  \begin{tabular}{|l|cr|l|} \hline
    Alignement à gauche & Données centrées & et à droite & Largeur imposée\\ \cline{2-3}
    Donnée & \multicolumn{2}{|c|}{Donnée} & \\ \hline
    1.10 & 2.20 & 3.20 & 4.69 \\
    1.10 & 2.20 & 3.20 & 4.69 \\
    1.10 & 2.20 & 3.20 & 4.69 \\ \hline
  \end{tabular}
\end{center}

%% Problèmes de tabular :

%% - contrôle de la largeur des cellules (array, tabularx)
%% - centrage vertical du texte dans la cellule (array)
%% - écartement entre le texte et les lignes horizontales (booktabs)

%% - tableaux longs à introduire sur plusieurs pages (longtable)

%% - multirow : regroupement vertical de cellules (doit-on vraiment l'aborder ?)


\section{Quelques problèmes centraux dans la composition d'un tableau}

\vfill

\subsection{Largeur des cellules}

Pour les colonnes contenant du texte de longueur importante, il faut
indiquer la largeur de colonne souhaitée, sinon l'on obtient quelque
chose d'assez désagréable : \LaTeX\ attribue à la cellule la largeur
du texte qu'elle contient (ce qui peut être vraiment grand !).

\subsection{Alignement vertical du texte}

L'environnement \emph{tabular} standard ne permet pas de contrôler
l'alignement vertical. Le texte est nécessairement aligné avec le haut
de la cellule.

\subsection{\'Ecartement texte / ligne horizontale}

Certaines lettres (les majuscules) touchent la ligne horizontale qui les domine.

\vfill


\section{Illustrations}

\subsection{Largeur des cellules}

\begin{boxedminipage}{\textwidth}
\begin{verbatim}
\begin{tabular}{|l|cr|c|} \hline
  Donnée & Donnée & Donnée & Ici un texte relativement long...\\ \hline
\end{tabular}
\end{verbatim}
\end{boxedminipage}

\begin{center}
  \begin{tabular}{|l|cr|c|} \hline
    Donnée & Donnée & Donnée & Sinon, \LaTeX\ adapte la taille de la colonne à la taille du texte. Ce qui peut poser des problèmes.\\ \hline
    % Donnée & Donnée & Donnée & \\ \hline
  \end{tabular}
\end{center}

\subsection{Alignement vertical}

\begin{tabular}{|p{.5\textwidth}|ccc|} \hline
  Texte relativement long blah blah blah blah blah blah blah blah blah blah blah blah blah blah blah blah blah blah & A & A & A \tabularnewline \cline{2-4}
  & A & A & A \tabularnewline \cline{2-4}
  & B & B & B \tabularnewline \hline
\end{tabular}


\subsection{Espacement caractère / ligne}

Le problème de l'espacement entre les caractères et la ligne
horizontale est manifeste dans ces deux exemples.



\section{Contrôle de la largeur des colonnes}

\vfill

On dispose de plusieurs solutions dont le choix dépend de la
situation et des préférences de l'utilisateur.


\begin{enumerate}
\item L'utilisateur souhaite laisser \LaTeX\ gérer lui-même la largeur des
  colonnes (gestion automatique) : utiliser l'extension \emph{tabularx}
\item L'utilisateur souhaite avoir le contrôle total de la largeur de chaque
  colonne : utiliser l'extension \emph{array}.
\end{enumerate}

Deux possibilités peuvent apparaître :
\begin{enumerate}
\item \label{toutes}Toutes les cases de la colonne vont contenir un texte de
  ce type. On utilise une technique de définition des colonnes avec l'appel à
  l'extension \emph{array}.
\item \label{titre}Seule une case (souvent dans la ligne de titre ou
  dans la colonne de gauche) contient un texte long, le reste
  contenant des données (par exemple numériques). On pourra utiliser
  soit \emph{array}, soit \emph{tabularx}.
\end{enumerate}

Pour le problème de l'espacement entre lignes horizontales et
caractères, on utilisera l'extension \emph{booktab}.

\vfill




% \section{Première situation}

% \begin{boxedminipage}{\textwidth}
% \begin{verbatim}
% \begin{center}
%   \begin{tabular}{|l|cr|p{.3\textwidth}|} \hline
%     Donnée & Donnée & Donnée & \begin{center}Et une case...\end{center}\\ \hline
%     ...
%     Donnée & \multicolumn{2}{|c|}{Donnée} & \\ \hline
%   \end{tabular}
% \end{center}
% \end{verbatim}
% \end{boxedminipage}


% \begin{center}
%   \begin{tabular}{|l|cr|p{.3\textwidth}|} \hline
%     Donnée & Donnée & Donnée & \begin{center}Et une case du tableau qui contient beaucoup de texte. La dimension de cette case est fixée, pas automatique.\end{center}\\ \hline
%     Donnée & Donnée & Donnée & \begin{center}Et une case du tableau qui contient beaucoup de texte. La dimension de cette case est fixée, pas automatique.\end{center}\\ \hline
%     Donnée & \multicolumn{2}{|c|}{Donnée} & \\ \hline
%   \end{tabular}
% \end{center}







% \section{Seconde situation}

%  \begin{boxedminipage}{\textwidth}
%  \begin{verbatim}
%  \begin{center}
%  \begin{tabular}{|l|cr|r|} \hline
%  Don. & Don. & Don. & \begin{minipage}{.4\textwidth} TEXTE \end{minipage}\\ \hline
%  1.10 & 2.20 & 3.20 & 4.69 \\
%  1.10 & 2.20 & 3.20 & 4.69 \\
%  1.10 & 2.20 & 3.20 & 4.69 \\ \hline
%  \end{tabular}
%  \end{center}
%  \end{verbatim}
%  \end{boxedminipage}

%  \begin{center}
%  \begin{tabular}{|l|cr|r|} \hline
%  Donnée & Donnée & Donnée & \begin{minipage}{.4\textwidth}Et une case du tableau qui contient beaucoup de texte. La dimension de cette case est fixée, pas automatique.\end{minipage}\\ \hline
%  1.10 & 2.20 & 3.20 & 4.69 \\
%  1.10 & 2.20 & 3.20 & 4.69 \\
%  1.10 & 2.20 & 3.20 & 4.69 \\ \hline
%  \end{tabular}
%  \end{center}

%  Quelques problèmes persistent :

%  \begin{itemize}
%  \item Comment joindre deux cellules verticalement ? (multirow)
%  \item Comment centrer le texte verticalement dans les cases ? (array)
%    \begin{itemize}
%    \item Remarquez qu'en utilisant minipage au lieu de \verb1p{largeur}1, les
%      autres cellules sont centrées verticalement.
%    \end{itemize}
%  \item Le texte des données chevauche les lignes horizontales du tableau
%    (booktab)
%  \end{itemize}







\section{L'extension \emph{tabularx}}

\vfill

\begin{itemize}
  
\item On appelle l'extension \emph{tabularx} et on utilise
  l'environnement \emph{tabularx}.

\begin{boxedminipage}{\textwidth}
\begin{verbatim}
\usepackage{tabularx}
...
\begin{tabularx}{.9\textwidth}{Xccc}
...
\end{tabularx}
\end{verbatim}
\end{boxedminipage}


\item On indique la largeur souhaitée pour tout le tableau (par
  exemple 90\% de la largeur de la page) et on marque les colonnes à
  faire varier automatiquement par un X. Les autres colonnes prennent
  la largeur du texte qu'elles contiennent (fonctionnement classique
  de \LaTeX).

\begin{tabularx}{.9\textwidth}{|X|ccc|} \hline
Texte relativement long blah blah blah blah blah blah blah blah blah blah blah blah blah blah blah blah blah blah & A & A & A \tabularnewline \cline{2-4}
& A & A & A \tabularnewline \cline{2-4}
& B & B & B \tabularnewline \hline
\end{tabularx}

\item Problèmes de tabularx : ne permet pas de contrôler l'alignement
  vertical du texte dans les cellules (toujours aligné en
  haut). L'apparence finale est --de mon point de vue-- nettement
  moins agréable qu'avec le contrôle fourni par \emph{array}.

\end{itemize}

\vfill




\section{L'extension \emph{array}}

\begin{itemize}
\item On appelle l'extension \emph{array} dans l'en-tête du document par
  \verb1\usepackage{array}1.
\item L'utilisation est ensuite exactement la même que dans la version standard
  de \LaTeX. (environnement tabular).
\item Par exemple :

  \begin{boxedminipage}{\textwidth}
\small
\begin{verbatim}
\begin{tabular}{b{.5\textwidth}b{.1\textwidth}cc} \hline
Texte relativement long blah ... & A & A & A \\ \cline{2-4}
                            & A & A & A \\ \cline{2-4}
                            & B & B & B \\ \hline
\end{tabular}
\end{verbatim}
  \end{boxedminipage}

\item donne :

\begin{tabular}{b{.5\textwidth}b{.1\textwidth}cc} \hline
Texte relativement long blah blah blah blah blah blah blah blah blah blah blah blah blah blah blah blah blah blah & A & A & A \tabularnewline \cline{2-4}
& A & A & A \\ \cline{2-4}
& B & B & B \\ \hline
\end{tabular}

\item Chaque colonne se voit attribuer une largeur fixée par le
  rédacteur. Le texte qui est plus large que cette colonne est
  \emph{plié} aux dimensions de la colonne. Vous remarquez que les
  déclarations des colonnes sont, pour certaines, très différentes :
  on a remplacé une lettre (c, l ou r) par la commande
  \verb1b{.5\textwidth}1 par exemple pour la première colonne. Les
  colonnes déclarées c, l ou r continuent d'être gérées classiquement
  (largeur automatique attribuée par \LaTeX).

\end{itemize}

\section{Améliorer l'usage de l'extension \emph{array}}

\begin{itemize}
\item \emph{array} fournit la possibilité de définir \emph{au préalable} des
  déclarations pour les colonnes.
\item L'instruction \verb1\newcolumntype{lettre}{définition}1 permet, avant la
  composition du tableau, de définir les caractéristiques des colonnes en leur
  attribuant des types.
\item Exemple :
\begin{verbatim}
\newcolumntype{t}{b{.40\textwidth}}
\newcolumntype{d}{b{.10\textwidth}}
\end{verbatim}
\item Définit 2 nouveaux types de colonnes : t (40\% de la largeur du
  texte, texte aligné sur le bas de la cellule) et d (10\% de la
  largeur du texte et texte également aligné sur le bas de la
  cellule).
\item On peut alors utiliser les lettres \emph{t} et \emph{d} à la place (ou en
  alternance) des lettres \emph{l}, \emph{c} ou \emph{r}.
\item Cette fonctionnalité permet même de modifier d'autres caractéristiques des
  cellules : notamment la police utilisée, la taille\ldots
\begin{verbatim}
\newcolumntype{t}{>{\slshape}b{.40\textwidth}}
\newcolumntype{d}{>{\centering\ttfamily\small}b{.10\textwidth}}
\end{verbatim}
\item Attention cependant : remplacer \verb1\\1 par \verb1\tabularnewline1 pour
  marquer les fins de lignes du tableau.
\end{itemize}




\section{Règle d'utilisation de \emph{newcolumntype}}

\begin{center}
\begin{verbatim}
    \newcolumntype{t}{>{formatage}b{.40\textwidth}}
\end{verbatim}
\end{center}

\begin{itemize}
\item Où le formatage peut servir à déclarer un choix de type de police :
  \begin{itemize}
  \item \verb1\rmfamily1 : Roman (= Serif, cf. Times)
  \item \verb1\sffamily1 : Sans Serif (cf. Helvetica)
  \item \verb1\ttfamily1 : Typewriter (cf. Courier) \emph{Particulièrement utile
      pour les colonnes de chiffres}
  \end{itemize}
\item une orientation des caractères :
  \begin{itemize}
  \item \verb1\slshape1 : Slanted (italique)
  \item \verb1\upshape1 : Droit (par défaut)
  \item \verb1\scshape1 : Small Caps (petites capitales)
  \end{itemize}
\item une graisse :
  \begin{itemize}
  \item \verb1\bfseries1 : Gras
  \end{itemize}
\item un alignement horizontal :
  \begin{itemize}
  \item \verb1\centering1 : Centré
  \item \verb1\raggedleft1 : Repoussé depuis la gauche (Aligné à droite)
  \item \verb1\raggedright1 : Repoussé depuis la droite (Aligné à gauche)
  \end{itemize}
\item une taille de caractères :
  \begin{itemize}
  \item \verb1\tiny1 : du plus petit
  \item \verb1\scriptsize1 : \ldots
  \item \verb1\footnotesize1 : \ldots
  \item \verb1\small1 : \ldots
  \item \verb1\normalsize1 : \ldots
  \item \verb1\large1 : \ldots
  \item \verb1\Large1 : \ldots
  \item \verb1\LARGE1 : \ldots
  \item \verb1\huge1 : \ldots
  \item \verb1\Huge1 : au plus grand
  \end{itemize}
\end{itemize}

Exemple : \verb2\newcolumntype{d}{>{\ttfamily\raggedleft}b{.1\textwidth}}2

Ces déclarations sont des commandes standard de \LaTeX\ (on peut les
utiliser ailleurs pour changer les caractéristiques d'un texte par
exemple).

\section{Un exemple plus complet : alignement sur le bas de la cellule}

\begin{boxedminipage}{\textwidth}
\begin{verbatim}
\newcolumntype{t}{>{\slshape\centering}b{.40\textwidth}}
\newcolumntype{d}{>{\ttfamily\raggedleft}p{.10\textwidth}}

\begin{center}
\begin{tabular}{tddd} \hline
Texte relativement long blah... & A & A & A \tabularnewline \cline{2-4}
        & A & A & A \tabularnewline \cline{2-4}
        & B & B & B \tabularnewline \hline
\end{tabular}
\end{center}
\end{verbatim}
\end{boxedminipage}

\newcolumntype{t}{>{\slshape\centering}b{.40\textwidth}}
\newcolumntype{d}{>{\ttfamily\raggedleft}p{.10\textwidth}}

\begin{center}
\begin{tabular}{tddd} \hline
Texte relativement long blah blah blah blah blah blah blah blah blah blah blah blah blah blah blah blah blah blah & A & A & A \tabularnewline \cline{2-4}
& A & A & A \tabularnewline \cline{2-4}
& B & B & B \tabularnewline \hline
\end{tabular}
\end{center}



\section{Un exemple plus complet : alignement sur le milieu vertical
  de la cellule}

\begin{boxedminipage}{\textwidth}
\begin{verbatim}
\newcolumntype{t}{>{\slshape\centering}m{.40\textwidth}}
\newcolumntype{d}{>{\ttfamily\raggedleft}p{.10\textwidth}}

\begin{center}
\begin{tabular}{tddd} \hline
Texte relativement long blah... & A & A & A \tabularnewline \cline{2-4}
        & A & A & A \tabularnewline \cline{2-4}
        & B & B & B \tabularnewline \hline
\end{tabular}
\end{center}
\end{verbatim}
\end{boxedminipage}

\newcolumntype{t}{>{\slshape\centering}m{.40\textwidth}}
\newcolumntype{d}{>{\ttfamily\raggedleft}p{.10\textwidth}}

\begin{center}
\begin{tabular}{tddd} \hline
Texte relativement long blah blah blah blah blah blah blah blah blah blah blah blah blah blah blah blah blah blah & A & A & A \tabularnewline \cline{2-4}
& A & A & A \tabularnewline \cline{2-4}
& B & B & B \tabularnewline \hline
\end{tabular}
\end{center}







\section{Notes sur l'utilisation de l'extension \emph{array}}

\vfill{}

\begin{itemize}
\item Attention : la largeur de la colonne est en réalité la largeur du texte
  contenu dans la colonne (non compris les espaces gauche et droit autour du
  texte). On risque donc de dépasser la largeur du texte si la somme des
  largeurs de colonnes est égale à 100\% (il suffit de le savoir).
\item L'usage de l'instruction \verb2b{.1\textwidth}2 indique que la colonne
  doit avoir une largeur équivalente à 10\% de la largeur totale du texte sur la
  page \emph{et} que ce texte doit être aligné sur la bas de la cellule (b =
  bottom). On peut aussi centrer le texte verticalement (remplacer \emph{b} par
  \emph{m}).  \emph{p} produit le fonctionnement standard (alignement sur le
  haut de la cellule).
\item La déclaration du type d'alignement vertical est attribuée à une
  colonne mais, évidemment, son effet va porter sur les lignes ! Il
  faut le comprendre comme une indication du type d'alignement qu'on
  va obtenir du texte contenu dans les autres colonnes par rapport à
  la cellule correspondante : si une cellule de la colonne 1 (type
  \verb1t1) contient beaucoup de texte, les autres cellules de la même
  ligne seront alignées sur le bas de cette première cellule. Dans les
  exemples précédents, si une cellule des 3 colonnes de gauche
  contenait beaucoup de texte, le texte des autres cellules serait
  aligné sur le haut de cette cellule !
 \item Noter qu'il est tout à fait possible d'alterner, dans un même document,
   l'usage des environnements \emph{tabular} et \emph{tabularx} en fonction des
   besoins.
\end{itemize}

\vfill{}



%% \section{L'extension \emph{multirow}}

%% \vfill

%% Utilisation de l'extension :

%% \begin{boxedminipage}{\textwidth}
%% \begin{verbatim}
%% \usepackage{multirow}

%% \begin{tabular}{lr} \hline
%% \multirow{nbrows}{largeur}[décalage]{Texte} & donnée \\
%%                                     & donnée \\ \hline
%% \end{tabular}
%% \end{verbatim}
%% \end{boxedminipage}

%% Nous verrons quel peut être l'usage de l'option [décalage] tout à l'heure. Pour
%% l'instant, nous souhaitons essentiellement voir comment écrire du texte dans
%% deux cellules (comment regrouper des cellules).

%% \vfill

%% \section{Utilisation de \emph{multirow}}


%% \begin{boxedminipage}{\textwidth}
%% \begin{verbatim}
%% \begin{center}
%% \begin{tabular}{|c|cp{.3\textwidth}|} \hline
%% \multirow{2}{*}{Titre commun} & Donnée & Un texte relativement ... \\ \cline{2-3}
%%                         & Donnée & Donnée \\ \hline
%% \multirow{2}{.2\textwidth}{Titre long} & Donnée & Un texte ... \\ \cline{2-3}
%%                         & Donnée & Donnée \\ \hline
%% \end{tabular}
%% \end{center}
%% \end{verbatim}
%% \end{boxedminipage}



%% \begin{center}
%% \begin{tabular}{|c|cp{.3\textwidth}|} \hline
%% \multirow{2}{*}{Titre commun} & Donnée & Un texte relativement long pour voir l'effet des cellules qui contiennent beaucoup de texte sur la mise en page du tableau. \\ \cline{2-3}
%%                         & Donnée & Donnée \\ \hline
%% \multirow{2}{.2\textwidth}{Titre commun qui contient plus de texte que n'en pourrait contenir une seule cellule, vraiment beacoup plus} & Donnée & Un texte relativement long pour voir l'effet des cellules qui contiennent beaucoup de texte sur la mise en page du tableau. \\ \cline{2-3}
%%                         & Donnée & Donnée \\ \hline
%% \end{tabular}
%% \end{center}

%% %% \begin{center}
%% %% \begin{tabular}{|c|cc|} \hline
%% %% \multirow{2}{*}{Titre commun} & Donnée & \begin{minipage}{.3\textwidth}Un texte relativement long pour voir l'effet des cellules qui contiennent beaucoup de texte sur la mise en page du tableau.\end{minipage} \\ \cline{2-3}
%% %%                         & Donnée & Donnée \\ \hline
%% %% \multirow{2}{.2\textwidth}{Titre commun qui contient plus de texte que n'en pourrait contenir une seule cellule} & Donnée & \begin{minipage}{.3\textwidth}Un texte relativement long pour voir l'effet des cellules qui contiennent beaucoup de texte sur la mise en page du tableau.\end{minipage} \\ \cline{2-3}
%% %%                         & Donnée & Donnée \\ \hline
%% %% \end{tabular}
%% %% \end{center}


%% Elle permet d'améliorer la mise en page du tableau, notamment lorsque celui-ci
%% contient des cellules qui doivent inclure une grande quantité de texte. Elle
%% utilise la même syntaxe que \emph{tabular}.

%% On l'appelle dans l'en-tête du document par \verb1\usepackage{array}1 et on
%% définit la largeur des cellules pour lesquelles on ne souhaite pas de contrôle
%% automatique en utilisant \verb1m{largeur}1 au lieu de \verb1p{largeur}1.

%% Pour faire intervenir \emph{array}, on définit la largeur d'au-moins une
%% cellule.


%% Avec \emph{array} (\verb1\begin{tabular}{|l|cr|m{.3\textwidth}|} \hline1) :
  
%%   \begin{boxedminipage}{\textwidth}
%% \begin{verbatim}
%% \begin{tabular}{|l|cr|m{.3\textwidth}|} \hline
%% Donnée & Donnée & Donnée & \begin{center}Et une case...\end{center}\\ \cline{2-3}
%% Donnée & Donnée & Donnée & \\ \hline
%% \end{tabular}
%% \end{verbatim}
%%   \end{boxedminipage}

%% \begin{center}
%% \begin{tabular}{|l|cr|m{.3\textwidth}|} \hline
%% Donnée & Donnée & Donnée & \begin{center}Et une case du tableau qui contient beaucoup de texte. La dimension de cette case est fixée, pas automatique.\end{center}\\ \cline{2-3}
%% Donnée & Donnée & Donnée & \\ \hline
%% \end{tabular}
%% \end{center}

%% \newpage

%% et sans \emph{array} (\verb1\begin{tabular}{|l|cr|p{.3\textwidth}|} \hline1) :

%%   \begin{boxedminipage}{\textwidth}
%% \begin{verbatim}
%% \begin{tabular}{|l|cr|p{.3\textwidth}|} \hline
%% Donnée & Donnée & Donnée & \begin{center}Et une case...\end{center}\\ \cline{2-3}
%% Donnée & Donnée & Donnée & \\ \hline
%% \end{tabular}
%% \end{verbatim}
%%   \end{boxedminipage}

%% \begin{center}
%% \begin{tabular}{|l|cr|p{.3\textwidth}|} \hline
%% Donnée & Donnée & Donnée & \begin{center}Et une case du tableau qui contient beaucoup de texte. La dimension de cette case est fixée, pas automatique.\end{center}\\ \cline{2-3}
%% Donnée & Donnée & Donnée & \\ \hline
%% \end{tabular}
%% \end{center}


\section{Le problème du chevauchement lettres / lignes}

Vous avez probablement remarqué que les lignes horizontales des tableaux
chevauchent le haut de certaines lettres.


\begin{itemize}
\item Pour empêcher les lettres de chevaucher les lignes horizontales du
  tableau, on peut introduire une instruction (à n'importe quel endroit du
  document mais en général dans l'en-tête):
  \begin{itemize}
  \item \verb1\setlength{\extrarowheight}{dimension}1
  \end{itemize}
\item Dans le tableau précédent, les lettres hautes (notamment les majuscules)
  touchent les lignes horizontales qui les dominent.
\item Si on ajoute \verb1\setlength{\extrarowheight}{0.3em}1, dans l'en-tête du
  document ou avant de composer le tableau, on obtient :
  \setlength{\extrarowheight}{0.3em}

\begin{center}
\begin{tabular}{|t|ddd|} \hline
Texte relativement long blah blah blah blah blah blah blah blah blah blah blah blah blah blah blah blah blah blah & A & A & A \tabularnewline \cline{2-4}
& A & A & A \tabularnewline \cline{2-4}
& B & B & B \tabularnewline \hline
\end{tabular}
\end{center}

% \item Mais cette procédure peut (lorsqu'on utilise des lignes
%   verticales, mais cf p.\ref{regles-table}) produire des vides.
\end{itemize}

Mais l'outil le plus intéressant pour construire les tableaux (en
combinaison avec \emph{array}) est l'extension \emph{booktabs},
développée pour construire des tableaux de qualité pour la publication
de documents écrits.


\section{L'extension \emph{booktabs}}
\label{booktab}
\setlength{\extrarowheight}{0ex}

\vfill
\begin{itemize}
\item \verb1\usepackage{booktabs}1
\item L'extension \emph{booktabs} s'occupe de formater un tableau en respectant
  scrupuleusement les nécessités typographiques en vigueur dans l'édition.
  \begin{itemize}
  \item Amélioration de la gestion des espaces entre les lignes (pas besoin de
    modifier arbitrairement l'espacement entre les lignes et le texte).
  \item Possibilité de faire varier l'épaisseur des lignes (épaisse en haut et
    en bas, fine à l'intérieur du tableau).
  \item Compatible avec \emph{array} (on peut fixer la largeur des colonnes et
    leur alignement vertical par exemple).
  \item Fonctionne aussi avec \emph{longtable} (cf. p.\ref{longtable}).
%  \item Permet d'insérer des sauts de lignes dans un texte relativement long.
  \end{itemize}
\item Notez qu'un tableau de données NE DOIT PAS comporter de lignes
  verticales.
\end{itemize}
\vfill
\newpage

\section{Différences entre \emph{array} et \emph{booktabs}}

\begin{itemize}
\item \emph{array} et \emph{booktabs} peuvent être utilisés simultanément.
\item Mais \emph{booktabs} doit être considéré comme une couche supplémentaire
  qui améliore la mise en page.
  
  \begin{center}
    \begin{tabular}{l!{$\rightarrow$}l} \toprule
      \emph{array} & \emph{booktabs} \tabularnewline \midrule
      \verb1\hline1 & \verb1\toprule1 \tabularnewline
      & \verb1\midrule1 \tabularnewline
      & \verb1\bottomrule1 \tabularnewline \midrule
      \verb1\cline{N-M}1 & \verb1\cmidrule{N-M}1 \tabularnewline \bottomrule
    \end{tabular}
  \end{center}
\end{itemize}

Notez que les épaisseurs de lignes peuvent ne pas apparaître
correctement à l'écran (visualisation PDF) mais apparaîtront
correctement à l'impression.


\section{Exemple d'utilisation de \emph{booktabs}}

\begin{boxedminipage}{\textwidth}
\begin{verbatim}
\usepackage{booktabs}

\begin{tabular}{lcrrr} \toprule
  & & Condition 1 & Condition 2 & Condition 3 \tabularnewline \midrule
  Expérience 1 & Temps de réaction (en ms) & 600 & 700 & 800 \tabularnewline
  & Taux d'erreur (en \%)  & 14 & 10 & 4 \tabularnewline \cmidrule{2-5}
  Expérience 2 & Temps de réaction (en ms) & 700 & 700 & 700 \tabularnewline
  & Taux d'erreur (en \%)  & 14 & 24 & 34 \tabularnewline \midrule
\end{tabular}
\end{verbatim}
\end{boxedminipage}

\begin{center}
  \begin{tabular}{lcrrr} \toprule
    & & Condition 1 & Condition 2 & Condition 3 \tabularnewline \midrule
    Expérience 1 & Temps de réaction (en ms) & 600 & 700 & 800 \tabularnewline
    & Taux d'erreur (en \%)  & 14 & 10 & 4 \tabularnewline \cmidrule{2-5}
    Expérience 2 & Temps de réaction (en ms) & 700 & 700 & 700 \tabularnewline
    & Taux d'erreur (en \%)  & 14 & 24 & 34 \tabularnewline \midrule
%%     Et même si l'on a \emph{vraiment} besoin d'insérer un texte relativement long dans cette cellule, son formatage est correct & Temps de réaction (en ms) & 700 & 700 & 700 \tabularnewline
%%     & Taux d'erreur (en \%)  & 14 & 24 & 34 \tabularnewline \bottomrule
%%     \multicolumn{3}{l}{\begin{minipage}{.6\textwidth}D'ailleurs un formatage avec extension de la cellule sur la ligne serait certainement plus agréable à lire. Même si le texte qu'elle contient est extrêmement long.\end{minipage}} & &  \\
%%     & Temps de réaction (en ms) & 700 & 700 & 700 \\
%%     & Taux d'erreur (en \%)  & 14 & 24 & 34 \\ \bottomrule
  \end{tabular}
\end{center}


\section{Et en collaboration avec \emph{array}\ldots}

\newcolumntype{t}{>{\slshape\raggedright}b{.20\textwidth}}
\newcolumntype{s}{>{\sffamily}c}
\newcolumntype{d}{>{\ttfamily}r}

\begin{boxedminipage}{\textwidth}
\begin{verbatim}
\newcolumntype{t}{>{\slshape\raggedright}b{.20\textwidth}}
\newcolumntype{s}{>{\sffamily}c}
\newcolumntype{d}{>{\ttfamily}r}

\begin{tabular}{lcrrr} \toprule
  & & Condition 1 & Condition 2 & Condition 3 \tabularnewline \midrule
  Expérience 1 & Temps de réaction (en ms) & 600 & 700 & 800 \tabularnewline
  & Taux d'erreur (en \%)  & 14 & 10 & 4 \tabularnewline \cmidrule{2-5}
  Expérience 2 & Temps de réaction (en ms) & 700 & 700 & 700 \tabularnewline
  & Taux d'erreur (en \%)  & 14 & 24 & 34 \tabularnewline \midrule
\end{tabular}
\end{verbatim}
\end{boxedminipage}

\begin{center}
  \begin{tabular}{tsddd} \toprule
    & & Condition 1 & Condition 2 & Condition 3 \tabularnewline \midrule
    Expérience 1 & Temps de réaction (en ms) & 600 & 700 & 800 \tabularnewline
    & Taux d'erreur (en \%)  & 14 & 10 & 4 \tabularnewline \cmidrule{2-5}
    Expérience 2 & Temps de réaction (en ms) & 700 & 700 & 700 \tabularnewline
    & Taux d'erreur (en \%)  & 14 & 24 & 34 \tabularnewline \midrule
%%     Et même si l'on a \emph{vraiment} besoin d'insérer un texte relativement long dans cette cellule, son formatage est correct & Temps de réaction (en ms) & 700 & 700 & 700 \tabularnewline
%%     & Taux d'erreur (en \%)  & 14 & 24 & 34 \tabularnewline \bottomrule
%%     \multicolumn{3}{l}{\begin{minipage}{.6\textwidth}D'ailleurs un formatage avec extension de la cellule sur la ligne serait certainement plus agréable à lire. Même si le texte qu'elle contient est extrêmement long.\end{minipage}} & &  \\
%%     & Temps de réaction (en ms) & 700 & 700 & 700 \\
%%     & Taux d'erreur (en \%)  & 14 & 24 & 34 \\ \bottomrule
  \end{tabular}
\end{center}


% \section{\ldots et avec \emph{tabularx}}

% \begin{boxedminipage}{\textwidth}
% \begin{verbatim}
% \newcolumntype{s}{>{\slshape\centering}m{.2\textwidth}}

% \begin{tabular}{Xsrrr} \toprule
%   & & Condition 1 & Condition 2 & Condition 3 \tabularnewline \midrule
%   Expérience 1 & Temps de réaction (en ms) & 600 & 700 & 800 \tabularnewline
%   & Taux d'erreur (en \%)  & 14 & 10 & 4 \tabularnewline \cmidrule{2-5}
%   Expérience 2 & Temps de réaction (en ms) & 700 & 700 & 700 \tabularnewline
%   & Taux d'erreur (en \%)  & 14 & 24 & 34 \tabularnewline \midrule
% \end{tabular}
% \end{verbatim}
% \end{boxedminipage}

% %\newcolumntype{s}{>{\slshape}c}
% \newcolumntype{s}{>{\slshape\centering}m{.3\textwidth}}
% \begin{center}
%   \begin{tabularx}{.9\textwidth}{Xsrrr} \toprule
%     & & Condition 1 & Condition 2 & Condition 3 \\ \midrule
%     Expérience 1 & Temps de réaction (en ms) & 600 & 700 & 800 \\
%     & Taux d'erreur (en \%)  & 14 & 10 & 4 \\ \cmidrule{2-5}
%     Expérience 2 & Temps de réaction (en ms) & 700 & 700 & 700 \\
%     & Taux d'erreur (en \%)  & 14 & 24 & 34 \\ \midrule
%   \end{tabularx}
% \end{center}


\section{Rappels essentiels sur la composition d'un tableau de données}
\label{regles-table}
\vfill

La construction d'un tableau répond à des règles strictes qu'il convient de
suivre pour les publications de travaux de recherche. En voici quelques unes
comme bref rappel :

\begin{itemize}
\item Un tableau ne doit jamais contenir de lignes verticales !
  \emph{Jamais} !
\item Les lignes horizontales doivent servir à séparer des éléments
  qui sont extrêments différents les uns des autres. Inutile de
  séparer toutes les lignes du tableau par une ligne horizontale si
  certaines lignes peuvent être groupées ensemble.
\item De fait, deux lignes correspondant à des données qui partagent
  un même en-tête doivent obéir à une règle simple : elles ne sont
  séparées par aucune ligne. Il est inutile --voire gênant-- de
  regrouper des cellules verticalement ; le simple fait que leur
  en-tête soit unique permet d'identifier les lignes comme relevant
  d'une même catégorie.
\item Par ailleurs, il est toujours essentiel, lorsque l'on rencontre
  des problèmes de mise en page, de se poser la question de la
  légitimité typographique de ce que l'on souhaite faire dans un
  tableau.
\item Si vous n'y arrivez pas avec \LaTeX, c'est (peut-être)
  simplement parce que vous devriez changer votre choix\ldots
\item Un tableau doit être le plus succinct possible :
  \begin{itemize}
  \item limiter les textes longs au strict minimum, ceux-ci trouvant plutôt leur
    place dans le corps du texte ou dans le titre du tableau.
  \item Ne pas abuser des changements de police de caractères
    (homogénéité au moins à l'intérieur des colonnes, à part
    éventuellement pour la ligne d'en-tête et la distinction texte /
    nombres).
  \end{itemize}
\item Si vous respectez ces quelques règles, vous ne devriez pas perdre de temps
  dans la composition de vos tableaux.
\end{itemize}

\vfill


% \newpage

% \vfill

% \begin{center}
%   Bon courage !
% \end{center}

% \vfill

%% \section{Un tableau typographiquement correct est en réalité extrêmement simple}

%% \begin{boxedminipage}{\textwidth}
%% \begin{verbatim}
%%   \begin{tabular}{m{.2\textwidth}lrrr} \hline
%%     & & Condition 1 & Condition 2 & Condition 3 \\ \hline
%%     Expérience 1 & Temps de réaction (en ms) & 600 & 700 & 800 \\
%%     & Taux d'erreur (en \%)  & 14 & 10 & 4 \\ \hline
%%     Expérience 2 & Temps de réaction (en ms) & 700 & 700 & 700 \\
%%     & Taux d'erreur (en \%)  & 14 & 24 & 34 \\ \hline
%%     Texte long & Temps de réaction (en ms) & 700 & 700 & 700 \\
%%     & Taux d'erreur (en \%)  & 14 & 24 & 34 \\ \hline
%%   \end{tabular}
%% \end{verbatim}
%% \end{boxedminipage}

%% \begin{center}
%%   \begin{tabular}{m{.2