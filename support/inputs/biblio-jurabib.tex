
\section{Les objectifs de \emph{jurabib}}


L'extension jurabib permet notamment de gérer des systèmes
bibliographiques requérant une présentation des références
bibliographiques en notes de bas de page, laquelle est assortie de
termes particuliers lorsque la référence a déjà été citée (op. cit.,
idem, ibidem\ldots).

L'atout majeur de jurabib est qu'il vous permettra de respecter toutes
ces règles \emph{sans avoir besoin de savoir (1) si vous avez déjà
  cité une référence, (2) quel terme vous devez utiliser si la
  référence a déjà été citée\ldots)}


\section{En-tête du document}

Pour utiliser \emph{jurabib} on charge l'extension et on définit un
certain nombre de paramètres :


\begin{small}
\begin{verbatim}
\usepackage{jurabib}
\jurabibsetup{%
  human=true,
  authorformat={abbrv,and,year,smallcaps},
  bibformat={ibidem},
  commabeforerest,
  ibidem=strict,
  idem=strict,
  opcit=true,
  citefull=first,
  biblikecite=true,
  oxford=true,
  titleformat={italic,commasep},
  annote=off,
}

\renewcommand{\bibauthormultiple}{--------\hspace{1em}}
\renewcommand{\cite}{\footcite}
\renewcommand{\bibbtsep}{in }
\renewcommand{\bibjtsep}{}
\renewcommand{\bibansep}{,}
\renewcommand{\bibatsep}{,}
\renewcommand{\bibbdsep}{}
\renewcommand{\bpubaddr}{~:}
\renewcommand{\bibjtfont}{\textit}
\end{verbatim}
\end{small}



\section{Appel des références bibliographiques}

Le fonctionnement est ensuite proche de celui de \emph{natbib} pour
les appels bibliographiques :


\begin{verbatim}

\cite{}
\citeauthor{}
\citeyear{}
\cite[p.15]{}

\end{verbatim}




\section{Formatage de la bibliographie}

Tout comme avec les autres systèmes, on indique le nom de la base
bibliographique et le style bibliographique (utilisez \emph{jox} qui
est fourni avec jurabib).

Si vous ne souhaitez pas qu'une bibliographie finale soit insérée dans
votre document, vous remplacerez \verb1\bibliography{}1 par
\verb1\nobibliography{}1.


\begin{verbatim}
\bibliography{filename} (ou \nobibliography{filename})
\bibliographystyle{jox}
\end{verbatim}


%%% Local Variables: 
%%% mode: latex
%%% TeX-master: "~/home/teaching/cours/sdl/M2/LaTeX/slides/recueil-initiation-LaTeX"
%%% End: 
