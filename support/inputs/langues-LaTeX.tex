

%%%%%%%%%%%%%%%%%%%%%%%%%%%%%%%%%%%%%%%%%%%%%%%%%%%%%%%%%%%%%%%%%%%%%%%%%%%

%%  LaTeX et le français

%%%%%%%%%%%%%%%%%%%%%%%%%%%%%%%%%%%%%%%%%%%%%%%%%%%%%%%%%%%%%%%%%%%%%%%%%%%



\section{\LaTeX\ et les spécificités typographiques}

\begin{itemize}
\item Dans chaque pays, il existe un ensemble de règles typographiques qui ont
  pour objet de réglementer la mise en page des documents écrits.
\item Ces règles sont plus ou moins générales :
  \begin{itemize}
  \item Certaines s'appliquent à tous les supports écrits (le traitement des
    espaces, l'interligne de début de paragraphe, l'indentation de début de
    paragraphe, la césure\ldots).
  \item D'autres sont spécifiques à un type de document (c'est notamment le cas
    des lettres dont la mise en page relève de règles bien particulières en
    fonction des pays).
  \end{itemize}
\end{itemize}


\section{Les espaces et le français}

\begin{itemize}
\item Tout le monde connaît les variations typographiques de base qui sont
  respectivement en usage dans les pays anglophones et francophones.
  \begin{itemize}
  \item Langue anglaise : pas d'espace avant les caractères de ponctuation.
  \item Langue française : espace\footnote{En réalité, les règles
      typographiques françaises imposent un espace d'une taille
      différente de l'espace entre deux mots.} avant les caractères
    \emph{doubles} (\emph{:},\emph{;},\emph{!},\emph{?}) mais pas
    devant les caractères simples.
  \end{itemize}
\item Il est très facile avec \LaTeX\ de gérer ces règles typographiques de
  manière transparente pour l'utilisateur. Sans aller trop loin dans l'étude de
  l'extension \LaTeX\ qui permet de gérer ce problème, nous allons commencer à
  l'utiliser dès maintenant. Ce qui nous permettra d'apprécier la qualité de la
  mise en page quelle que soit la langue utilisée.
\end{itemize}

%\newpage 

Nous allons donc introduire dans l'en-tête du document (avant
\verb1\begin{document}1) l'appel :
  
\begin{flushright}
  \verb1\usepackage[french]{babel}1.
\end{flushright}
  
\begin{exemple}[H]
  \caption{Usage des règles typographiques du français avec \LaTeX}
\begin{verbatim}
\documentclass[a4paper]{article}
\usepackage[utf8]{inputenc}
\usepackage{xspace}
\usepackage[english,french]{babel}

\begin{document}
...
\end{document}
\end{verbatim}
\end{exemple}


Cette extension va notamment avoir pour fonction :

\begin{itemize}
\item D'insérer un espace insécable \emph{avant} les signes de
  ponctuation qui le nécessitent (que vous mettiez vous-même un espace
  ou pas, \emph{babel} se chargera de faire ce qu'il faut pour que la
  taille de l'espace soit toujours la même).
\item D'empêcher les sauts de lignes avant ces espaces (plus de caractères de
  ponctuation en début de ligne) sans que vous ayez quoi que ce soit à faire.
\item De couper (en anglais \emph{hyphenation}, \emph{césure}) les
  mots français où il le faut pour que la mise en page soit agréable.
\item De franciser votre document (la date sera écrite en français notamment).
\item \ldots (nous étudierons l'extension \emph{babel} plus en détails
  par la suite).
\end{itemize}



\section{Francisation automatique du texte}

\begin{itemize}
\item En appelant l'extension \emph{Babel}, les mots qui sont insérés
  automatiquement par \LaTeX\ seront automatiquement francisés :
  \begin{itemize}
  \item bibliography vs. bibliographie
  \item chapter vs. chapitre
  \item Abstract vs. Résumé
  \item Appendix vs. Annexe
  \item Table of contents vs. Table des matières
  \item List of figures vs. Liste des figures
  \item francisation de la date
  \item francisation des titres de figures / tableaux
  \item \ldots
  \end{itemize}
\end{itemize}



\section{Autres paramètres modifiés par l'appel à Babel}

\begin{itemize}
\item modification des \og bullets \fg devant les listes
\item réglage de l'espacement entre les éléments d'une liste
\item permet d'utiliser les instructions \verb1\og1 et \verb1\fg1 respectivement
  pour \og Ouvrir les Guillemets français \fg (og) et \og Fermer les Guillemets
  français \fg (fg).% (ou \verb1<<1 et \verb1>>1).
\item On peut obtenir les guillemets \emph{anglais} avec \verb1``1 et \verb1''1,
  ce qui donne `` et ''. Si vous utilisez Emacs, il convertit automatiquement
  \verb1"1 en guillemets anglais ouvrants ou fermants en fonction de leur
  position par rapport au mot : Si vous tapez \verb1"1 après un espace, il les
  convertit en guillemets ouvrants\ldots sinon en guillemets fermants.
\end{itemize}


\section{Basculer d'une langue à l'autre}

\begin{exemple}[H]
  \caption{Usage des règles typographiques du français avec \LaTeX}
\begin{verbatim}
\documentclass[a4paper]{article}
\usepackage[utf8]{inputenc}
\usepackage{xspace}
\usepackage[english,french]{babel}

\begin{document}
...
\selectlanguage{english}
...
\selectlanguage{french}
...
\selectlanguage{english}
...
\end{document}
\end{verbatim}
\end{exemple}

On utilise aussi l'extension \emph{xspace} qui améliore la gestion des
espaces (par exemple dans le cas des guillemets).

Par défaut, la langue par défaut du document est la dernière langue
déclarée dans l'appel à usepackage.


\section{Rédiger des documents dans d'autres langues}
\label{sec:babel}

Ceci n'est qu'un petit aperçu des possibilités de Babel. Vous pouvez
également rédiger des documents dans bien d'autres langues en
utilisant babel (allemand, espagnol, grec, breton, basque, hébreu,
Hongrois, Russe, Turc\ldots). Pour celà, n'hésitez pas à lire la
documentation de Babel:

\bibentry{babeldoc}.

Comme vous pourrez le remarquer, Babel fonctionne surtout pour des
langues qui s'écrivent de gauche à droite. Pour l'arabe (ainsi que
l'hébreu), on utilise en général l'extension \verb1arabtex1. Pour les
langues comme le chinois, le coréen, le japonais, on utilise
l'extension \verb1CJK1.

N'hésitez pas à chercher sur internet s'il existe une extension qui
soit adaptée à votre langue maternelle ou autre (il existe par exemple
des extensions pour écrire avec les systèmes d'écriture
hiéroglyphique, copte, éthiopien\ldots) et à m'en faire part
(\url{mailto:olivier.crouzet@univ-nantes.fr}) si vous trouvez quelque
chose d'intéressant.

Vous pouvez par exemple consulter la page internet suivante :

\url{http://tug.ctan.org/tex-archive/language/}






